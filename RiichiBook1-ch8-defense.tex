%~~~~~~~~~~~~~~~~~~~~~~~~~~~~~~~~~~~~~~~~~~~~~~~~~
% Riichi Book 1, Chapter 8: Defense
%~~~~~~~~~~~~~~~~~~~~~~~~~~~~~~~~~~~~~~~~~~~~~~~~~

\chapter{Defense judgement} \label{ch:defense}
\thispagestyle{fancy}

\section{To push or to fold?}
Knowing when to push and when to fold is another important element of mahjong strategies. Push--fold judgement is a lot more complicated than {\jap riichi} judgement covered in the previous chapter. 
In presenting defense strategies, I will first describe a very simple principle that tells you when to be defensive and when to be offensive, based purely on your hand. After understanding this principle, the next step is to understand \emph{how} to be defensive. The latter part of this chapter introduces a set of defensive techniques. 

\bigskip
\subsection{A simple principle}
A lot of variables can factor into our decision to push or to fold against the opponents. You may want to consider, among other things, whether or not you currently have a ready hand, the potential hand value of your hand, the likely hand value of an opponent's hand, your current rank in the game, the opponent's standing in the game, just to name a few. 

\bigskip
It is simply impossible to take into account these and other important factors all at once in a limited amount of time. 
Instead, I suggest you utilize the following shortcut for push/fold judgement.

\begin{itembox}[c]{Push/fold judgement}
When another player has a ready hand,\\
\\
{\large Push} if {\bf two} of the following conditions are met:
\vspace{-10pt}
\be\itemsep.2em
\i Ready hand;
\i High scoring hand;
\i Good wait.
\ee
\vsps
{\large Fold} if {\bf two} of the following conditions are met:
\vspace{-10pt}
\be\itemsep.2em
\i 1-away (or worse) hand;
\i Low scoring hand;
\i Bad wait.
\ee
\end{itembox}

\noindent Let me explain each component of this principle in turn. 

\subsection{Guessing if an opponent has a ready hand}
First, you need to guess if another player has a ready hand or not; if your opponent does not have a ready hand, there is no point in playing defensive. Of course, knowing whether an opponent has a ready hand can be difficult. Rather than spending too much time trying to guess if they have a ready hand, let's stick with rough but simple shortcuts. 

There are three possibilities to consider. 
\bi
\i[A.] {\jap Riichi}\\
	This is the easiest case. You can be fairly certain that the opponent has a ready hand. We will talk about how to defend against {\jap riichi} in Section \ref{sec:defense_riichi}.
\i[B.] Melded ready hand\\
	Knowing whether or not an opponent has a melded ready hand is a bit complicated. We will discuss this in Section \ref{sec:defense_meld}. 
\i[C.] {\jap Dama} ready hand\\
	We will completely ignore this case.
\ei

\bigskip
Assuming an opponent would not have a {\jap dama} ready hand is obviously not always correct. Nevertheless, this shortcut would be acceptable given that accurately guessing whether or not an opponent has a ready hand is extremely difficult. Part of the reason why it is OK to ignore the case of {\jap dama} ready hand lies in the fact that {\jap riichi} is such a powerful tool in Riichi Mahjong that calling {\jap riichi} is strictly better than going {\jap dama} in most instances; your opponents cannot win a game if they keep choosing {\jap dama} when they should call {\jap riichi} (and they are likely to know it). 

\subsection{Three conditions to push/fold}
Note that, according to the principle laid out above, just because (you think) an opponent has a ready hand, it does not automatically mean that you must fold immediately. Specifically, you should still push if two out of the three conditions  specified above --- ready, high score, and good wait --- are met. 

\bigskip
The first condition is fairly straightforward. Just remember that a clear, firm line should be drawn between having a ready hand and having a 1-away (or worse) hand. Pushing with a 1-away hand is acceptable only when \emph{both} of the other two conditions are met. 

\bigskip
The second condition is also straightforward. We say a hand is a high scoring one if the minimum hand value is 7700; otherwise it is a low scoring hand (recall the discussion in Section \ref{sec:high} from the previous chapter). 

\bigskip
The third condition (good / bad waits) needs some explanation. When you have a ready hand, this is straightforward. You can decide if the third condition is met simply based on the waits classification discussed in Section \ref{sec:badwaits} from the previous chapter. That is, wait is good if it is at least as good as stretched single or semi side wait (2 kinds--6 tiles); pair wait, closed wait, edge wait, and single wait are usually considered to be a bad wait. 

\bigskip
The question then is: how do we judge if a hand has a good wait when the hand is 1-away or worse? 
When a hand is 1-away or worse, your judgement should be based on the \emph{best} possible wait pattern you can choose when the hand becomes ready in the \emph{worst} possible manner. An example will be helpful. Suppose you have the following hand when another player calls {\jap riichi}. 
\bp
\wan{2}\wan{3}\wan{4}\wan{6}\wan{6}\tong{6}\tong{6}\tong{7}\suo{4}\suo{5}\fa\fa\fa\bei~~\fa\\
\hspace{315pt}\footnotesize{\jap Dora}
\ep \index{1-away (1-{\jap shanten})!perfect 1-away}
This hand is not ready but has the minimum of 7700 point, so your decision to push or fold depends on the third condition. Notice that this is a perfect 1-away hand; no matter how this hand becomes ready, you can \emph{always} choose to have a side wait. In other words, the \emph{best} possible wait pattern you can choose when this hand becomes ready is always a good one. If you draw or {\jap pon} on a {\large \wan{6}} or a {\large \tong{6}} or {\jap chii} {\large \tong{5}-\tong{8}}, you can have a side wait of {\large\suo{3}-\suo{6}}. If you draw or {\jap chii} {\large \suo{3}-\suo{6}}, you can have a side wait of {\large\tong{5}-\tong{8}}. Therefore, you can push to the fullest ({\jap zentsu}) with this hand.

\bigskip
On the other hand, the following hand is also 1-away from ready with a high scoring potential, but it will not always lead to a good-wait ready hand. 
\vspace{-25pt}
\bp 
\hspace{-145pt}{\footnotesize\color{red!75!black} Red}\\ \vspace{-18pt}
\wan{3}\wan{4}\tong{3}\tong{4}\rfd\tong{6}\tong{8}\tong{9}\suo{2}\suo{3}\suo{4}\suo{6}\suo{6}\bei~~\suo{3}\\
\hspace{315pt}\footnotesize{\jap Dora}
\ep
Specifically, it will be a side-wait ready hand \emph{only} if you draw a {\large\tong{7}} first. If you draw a {\large\wan{2}-\wan{5}} first (which has a much higher probability than drawing a {\large\tong{7}}), it will be a closed-wait hand. Therefore, the \emph{best} possible wait pattern in the \emph{worst}-case scenario is not a good one. Therefore, you should fold with this hand when you are forestalled by opponents. 

\bigskip
For another example, consider a {\jap chiitoitsu} hand. A {\jap chiitoitsu} hand will always have a bad wait. This means that you should in principle fold if an opponent calls {\jap riichi} when your {\jap chiitoitsu} hand is not ready, even if you have two or more {\jap dora} in your hand.

\section{Defense basics} \label{sec:defense}

Once you understand the criteria to fold, the next thing you need to know is \emph{how} to fold. There are three main ways to identify safe tiles to discard. 

\subsection{{\jap Genbutsu} and other absolutely safe tiles} \index{dama@{\jap dama}} \label{sec:genbutsu_def}
I introduced the term {\jap genbutsu} in the previous Chapter. Strictly speaking, {\jap genbutsu} tiles of player X refers to those tiles discarded by X herself. However, if X has called {\jap riichi}, then all the tiles discarded by anyone after {\jap riichi} (and passed up by X) are also called X's {\jap genbutsu} tiles. 

\bigskip
{\jap Genbutsu} tiles of player X are 100\% safe against X, but not necessarily safe against the other two players. There are three kinds of tiles that are 100\% safe against all of your opponents. 
\bi
\i The tile that was just discarded by the left player.
\i A fourth honor tile when there is no possibility of Thirteen Orphans.
\i An absolute ``no chance'' tile.
\ei
The first kind is fairly straightforward. Because of the {\jap furiten} rule, the tile just discarded by the left player is 100\% safe for you to discard in the present turn. That tile is not only {\jap genbutsu} for the left player but also a temporary {\jap genbustu} for the right and the facing players. Until their temporary {\jap furiten} status is lifted, the right and the facing players cannot call {\jap ron} on it. 

\bigskip
The second kind of absolutely safe tile is relatively simple. Suppose all four of {\large\tong{9}} are visible to you (among the discards or in your hand). Then, none of your opponents can win Thirteen Orphans unless you discard a {\large\tong{9}}. In such situations, a fourth honor tile is 100\% safe. That is, a {\large\dong} is 100\% safe for everyone if all the other three tiles of {\large\dong} are visible to you.

\bigskip
The third kind, absolute ``no chance'', needs some explanation. Let me just give you an example here. Suppose all four tiles of {\large \tong{2}}, all four tiles of {\large \tong{4}}, and all three tiles of {\large \tong{3}} are visible to you. Then, the fourth {\large \tong{3}} is 100\% safe for everyone because this tile cannot be a part of any set, run, or pair. I will explain more about ``no chance'' tiles in Section \ref{sec:blockade} of this chapter.

\bigskip
Of course, it is not always possible to find tiles that are 100\% safe for the player who has called {\jap riichi} (let alone for all three opponents). Therefore, we need to know how to identify relatively safe tiles by relying on {\jap suji} and {\jap kabe} (blockade) theories. I will introduce these two theories in turn.

\subsection{Understanding {\jap suji}} \index{suji@{\jap suji}} \label{sec:suji}

When someone calls {\jap riichi}, the possibility you need to be wary of first and foremost is that the opponent has a side-wait hand. It is true that players will call {\jap riichi} even when wait is worse than side wait. However, according to some statistics, about two thirds of {\jap riichi} hands have a side wait or better. This is partly because the likelihood of choosing {\jap dama} increases when the wait is bad. Another reason is that players seek to retain side-wait protoruns over closed- or edge-wait protoruns when choosing tile blocks to maximize tile efficiency.

\bigskip
{\jap Suji} defense is a defense tactics to avoid dealing into a side-wait hand. A {\jap suji} is a three-tile interval that corresponds to the wait of a side-wait hand. For example, when a hand has a side-wait protorun {\large \wan{2}\wan{3}}, the wait is {\large \wan{1}} or {\large \wan{4}}. This wait combination of {\large\wan{1}} and {\large\wan{4}} is called {\large\wan{1}-\wan{4}} {\jap suji}. There are 6 {\jap suji} in each suit, giving rise to 18 {\jap suji} in total. All the six {\jap suji} and their corresponding side-wait protoruns are summarized in Table \ref{tbl:suji}. 

{\begin{table}[h!]\centering \small\captionsetup{font=small}
\caption{Six {\jap suji}} \label{tbl:suji}
\begin{tabular}{c c c c}
\toprule
{\jap suji} & protorun & {\jap suji} & protorun\\
\midrule
1-4 {\jap suji} & {\Large \wan{2}\wan{3}} &4-7 {\jap suji} & {\Large \wan{5}\wan{6}}\\
2-5 {\jap suji} & {\Large \wan{3}\wan{4}} &5-8 {\jap suji} & {\Large \wan{6}\wan{7}}\\
3-6 {\jap suji} & {\Large \wan{4}\wan{5}} &6-9 {\jap suji} & {\Large \wan{7}\wan{8}}\\
\bottomrule
\end{tabular}
\end{table}

\bigskip
When a {\large\wan{4}} is among a player's {\jap genbutsu}, we say {\large\wan{1}} and {\large\wan{7}} are {\jap suji} tiles. {\jap Suji} tiles are safer than non-{\jap suji} tiles because the {\jap furiten} rule prohibits a player from calling {\jap ron} on a {\large\wan{1}} when a {\large\wan{4}} is her {\jap genbutsu} \emph{and} her wait is {\large\wan{1}-\wan{4}}. 
Likewise, when a {\large\wan{5}} is among a player's {\jap genbutsu}, {\large\wan{2}} and {\large\wan{8}} are {\jap suji} tiles and thus they are safer than non-{\jap suji} tiles; when a {\large\wan{6}} is among a player's {\jap genbutsu}, {\large\wan{3}} and {\large\wan{9}} are {\jap suji} tiles and thus they are safer than non-{\jap suji} tiles. 

\bigskip
Although {\large\wan{4}} makes {\large\wan{1}} a {\jap suji} tile, the opposite is not true; {\large\wan{1}} in itself does \emph{not} make {\large\wan{4}} a {\jap suji} tile. {\large\wan{4}} is no safer than other tiles just because a {\large\wan{1}} is among {\jap genbutsu}. 
What {\large\wan{1}} negates is the possibility of the {\large\wan{1}-\wan{4}} side wait, but the {\large\wan{4}-\wan{7}} side wait is still a possibility. {\large\wan{4}} becomes safer only when \emph{both} a {\large\wan{1}} and a {\large\wan{7}} are among a player's {\jap genbutsu}. Table \ref{tbl:sujic} summarizes combinations of {\jap genbutsu} tiles and tiles that are made safer by them.

{\begin{table}[t!]\centering \small\captionsetup{font=small}
\caption{{\jap suji} tiles} \label{tbl:sujic} 
\begin{tabular}{l r}
\toprule
{\jap Genbutsu} & {\jap suji} tiles\\
\midrule
{\Large\wan{4}} discarded & {\Large\wan{1}} and {\Large\wan{7}} become safer\\
{\Large\wan{5}} discarded & {\Large\wan{2}} and {\Large\wan{8}} become safer\\
{\Large\wan{6}} discarded & {\Large\wan{3}} and {\Large\wan{9}} become safer\\
{\Large\wan{1}} and {\Large\wan{7}} discarded & {\Large\wan{4}} becomes safer\\
{\Large\wan{2}} and {\Large\wan{8}} discarded & {\Large\wan{5}} becomes safer\\
{\Large\wan{3}} and {\Large\wan{9}} discarded & {\Large\wan{6}} becomes safer\\
\bottomrule
\end{tabular}
\end{table}

\subsubsection*{{\jap Suji} trap}
	\index{suji@{\jap suji}!{\jap suji} trap}
Keep in mind that {\jap suji} defense works only against side-wait hands. Since players will call {\jap riichi} even when their wait is worse than side wait, we cannot rely too much on {\jap suji}. 
When you wait for a tile that is a {\jap suji} tile of some tiles you have discarded yourself, we say you have a {\jap suji}-trap wait. In particular, when your wait is a {\jap suji} tile of the {\jap riichi} declaration tile, we say it is an immediate {\jap suji}-trap wait. An immediate {\jap suji}-trap {\jap riichi} is a rather common occurrence in Riichi Mahjong primarily because of double closed block (e.g., 135, 246, 357, etc.). Consider the following hand.
\bp
\wan{2}\wan{4}\wan{6}\wan{9}\wan{9}\wan{9}\tong{6}\tong{6}\suo{3}\suo{4}\suo{5}\fa\fa\fa~~\suo{3}\\
\hspace{315pt}\footnotesize{\jap Dora}
\ep
If you call {\jap riichi} by discarding the {\large\wan{6}}, the hand waits for {\large\wan{3}}, which is a {\jap suji} tile of {\large\wan{6}} (immediate {\jap suji}-trap {\jap riichi}). 

\bigskip
In general, the reliability of {\jap suji} is higher for tiles discarded earlier in a hand. That is, {\jap suji} tiles of early discards tend to be safer, whereas {\jap suji} tiles of those tiles that are discarded later are more dangerous. In particular, {\jap suji} tiles of the tile discarded upon {\jap riichi} is at least as dangerous as non-{\jap suji} tiles.\footnote{You might wonder about the reliability of {\jap suji} for those tiles discarded \emph{after} {\jap riichi}. There is in fact a big disagreement among professional players about whether those tiles discarded after {\jap riichi} make for safe or dangerous {\jap suji}. For example, 渋川 難波 (Nanba Shibukawa; NPM) argues that {\jap suji} tiles of those discarded after {\jap riichi} are more dangerous than {\jap suji} tiles of those discarded before {\jap riichi}, whereas 石橋 伸洋 (Nobuhiro Ishibashi; {\jap Saikouisen}) argues the exact opposite. However, both schools of thought agree that the very tile discarded upon {\jap riichi} makes for dangerous {\jap suji}. See \url{http://osamuko.com/identifying-dangerous-suji/} for some data analyses.} \index{Osamuko}

\bigskip
For example, suppose an opponent calls {\jap riichi} in the 7th turn with the following discards.
\bp
\bei\fa\wan{9}\wan{4}\tong{2}\tong{6}\rsuo{5}\\
\hspace{130pt}\footnotesize{{\jap riichi}}
\ep

\newpage
There are three tiles in the discard that create {\jap suji} tiles.
\bi
\i The {\large\wan{4}} makes {\large\wan{1}} and {\large\wan{7}} {\jap suji} tiles.
\i The {\large\tong{6}} makes {\large\tong{3}} and {\large\tong{9}} {\jap suji} tiles.
\i The {\large\suo{5}} makes {\large\suo{2}} and {\large\suo{8}} {\jap suji} tiles.
\ei
Note that the {\large\wan{9}} and the {\large\tong{2}} do not create any {\jap suji} tiles on their own. Among these three tiles {\large\wan{4}\tong{6}\suo{5}}, {\jap suji} tiles of {\large\wan{4}} are relatively safe, whereas {\jap suji} tiles of {\large\suo{5}} are rather dangerous. 
The reason why {\jap suji} tiles of {\large\suo{5}} are dangerous is that players tend to keep double closed block such as {\large\suo{5}\suo{7}\suo{9}} or {\large\suo{1}\suo{3}\suo{5}} until the hand becomes ready. Consider the following 1-away hand. 
\bp
\hspace{65pt}{\footnotesize\color{red!75!black} Red}\\ \vspace{-18pt}
\wan{1}\wan{1}\wan{1}\wan{6}\wan{7}\wan{8}\tong{1}\tong{1}\rfd\tong{6}\tong{6}\suo{1}\suo{3}\suo{5}
\ep
We would discard a {\large\tong{6}} rather than the {\large\suo{5}}.
Then, if we draw {\large\tong{4}-\tong{7}} first, we do insta-{\jap riichi} by discarding the {\large\suo{5}}, creating an immediate {\jap suji}-trap wait. 


\subsection*{3. Understanding tile blockade ({\jap kabe})} \label{sec:blockade}
	\index{blockade@blockade ({\jap kabe}; wall)}
	\index{kabe@{\jap kabe} (blockade; wall)}
Another defense tactics to identify safer tiles is to utilize a {\bf tile blockade} ({\jap kabe}; wall). When chunks of a number tile are visible to you, we say these tiles form a blockade; they block a formation of runs that contain that tile. Suppose all four of {\large\wan{2}} are visible to you, either because they have been discarded or they are in your hand.
Then, none of your opponents can have a {\large\wan{1}-\wan{4}} {\jap suji} wait, making {\large\wan{1}} relatively safe. This is because no one can have a protorun {\large\wan{2}\wan{3}} in this situation. 

\subsubsection{No chance}
When all four of a number tile are visible, we say we have a ``no chance'' situation, meaning that there is no chance that an opponent has a {\jap suji} wait that contains the tile that forms a blockade. In the example above, {\large\wan{1}} is a no-chance tile thanks to a blockade of {\large\wan{2}}. 

\bigskip
No-chance tiles are safer than a non-{\jap suji} tile, but keep in mind that pair wait and single wait of a no-chance tile is still possible. 
A nice thing about no-chance tiles is that their safety does not depend on whether it is a {\jap suji} tile or note. For example, when we have a blockade of {\large\wan{2}}, {\large\wan{1}} is safe regardless of whether {\large\wan{4}} is among {\jap genbutsu}. Note also that {\large\wan{4}} is not necessarily safe just because we have a blockade of {\large\wan{2}}, for a {\large\wan{4}-\wan{7}} {\jap suji} wait is still a possibility. 
Of course, when all four of {\large\wan{2}} are visible to you \emph{and} a {\large\wan{7}} is among a player's {\jap genbutsu}, then {\large\wan{4}} becomes safer for that player. 

\begin{floatingtable}[r]{
\centering \small\captionsetup{font=footnotesize}
~~\begin{tabular}{c c}
\toprule
Blockade & Safe tiles\\
\midrule
{\Large\wan{1}} & None\\
{\Large\wan{2}} & {\Large\wan{1}}\\
{\Large\wan{3}} & {\Large\wan{1}\wan{2}}\\
{\Large\wan{4}} & {\Large\wan{2}\wan{3}}\\
{\Large\wan{5}} & {\Large\wan{3}\wan{7}}\\
{\Large\wan{6}} & {\Large\wan{7}\wan{8}}\\
{\Large\wan{7}} & {\Large\wan{8}\wan{9}}\\
{\Large\wan{8}} & {\Large\wan{9}}\\
{\Large\wan{9}} & None\\
\bottomrule
\end{tabular}}
\vspace{-10pt}
\caption{Blockades} \label{tbl:kabe} \vsp
\end{floatingtable}

\bigskip
Table \ref{tbl:kabe} summarizes possible blockades and the resulting \\no-chance safe tiles. 
Notice that each blockade can produce at most two sets of safe tiles. It should be easy to see how a blockade of 1 does not make any tile safer, a blockade of 2 makes 1 safer, and that a blockade of 3 makes 1 and 2 safer. 
However, a blockade of 4 makes only 2 and 3 safer, as it negates 2-5 and 3-6 waits that contain 4, but it does not make 1 any safer. Similarly, a blockade of 5 makes only 3 and 7 safer, negating 3-6 and 4-7 {\jap suji} waits. 

\bigskip
A blockade can negate non-{\jap suji} waits as well. For example, if all four of {\large\wan{3}} and all four of {\large\wan{5}} are both visible, then a closed wait of {\large\wan{4}} is impossible. An opponent has to have a pair wait or a single wait if she were to win on a {\large\wan{4}}. If, additionally, all three of {\large\wan{4}} are visible to you as well, then the fourth {\large\wan{4}} is 100\% safe. Waiting for a {\large\wan{4}} in this situation is simply impossible. 

\bigskip
A blockade can also negate certain {\jap yaku}, which decreases the chance that an opponent has an expensive hand. For example, when all four of {\large\wan{3}} are visible to you. Then an opponent cannot have {\jap ittsu} (Pure Straight) in Cracks. This information can help us decide whether to discard a non-{\jap suji} {\large\wan{8}} or a non-{\jap suji} {\large\suo{8}}. The chance of dealing into an opponent's hand is equal, but the chance of dealing into an expensive hand is lower with the {\large\wan{8}}. 
A blockade of {\large\wan{3}} also negates {\jap sanshoku} of 123, 234, or 345. This information can help us decide whether to discard a non-{\jap suji} {\large\suo{2}} or a non-{\jap suji} {\large\suo{8}}. Again, the chance of dealing into an opponent's hand is equal, but the chance of dealing into an expensive hand is lower with {\large\suo{2}}. 

\subsubsection*{One chance}

When only three of a number tile are visible to you, we have an incomplete blockade, making for what's called ``one chance'' tiles. One-chance tiles are generally safer than non-{\jap suji} tiles, but not as safe as no-chance tiles. 
The reliability of incomplete blockades depends on two things. 

\bigskip
First, relying on an incomplete blockade is effective in earlier turns but not as much in later turns. Suppose that an opponent calls {\jap riichi}, and a {\large\wan{2}} is among his discard. Then, because other players are likely to discard a {\large\wan{2}} if they have one, this tile may become an incomplete blockade later on. However, an incomplete blockade formed this way is not very reliable. When all three players are being defensive \emph{and} the fourth {\large\wan{2}} is still invisible, then it is highly likely that the {\jap riichi}-ed player has it. One-chance tiles would become almost as dangerous as non-{\jap suji} tiles in later turns in situations like this. 

\bigskip
Second, one-chance tiles are more reliable when the incomplete blockade that makes for a one-chance tile is known \emph{only to you}, thanks to a concealed set or a pair in your hand. On the other hand, one-chance tiles that are created by an incomplete blockade in the discard pool are not particularly safe. This is because an opponent is more likely to choose {\jap riichi} over {\jap dama} when one of her winning tiles is a one-chance tile and appears safe.

\bigskip

When we have two incomplete blockade of consecutive number tiles, we say they form a ``double one chance'' situation. For example, if three of {\large\wan{2}} and three of {\large\wan{3}} are both visible, an opponent has to have the fourth {\large\wan{2}} \emph{and} the fourth {\large\wan{3}} to have a {\large\wan{1}-\wan{4}} {\jap suji} wait, which is highly unlikely. Therefore, double-one-chance tiles are safer than a single one-chance tiles. 

\bigskip
\begin{itembox}[c]{Tile blockade: Safety ranking}
No chance $>$ Double one chance $>$ One chance (earlier turns) $>$ One chance (later turns) $\simeq$ Non-{\jap suji}
\vsps
\end{itembox}
\bigskip

\newpage
\subsubsection*{Combining blockade and {\jap suji}}

\begin{floatingtable}[r]{
\centering \footnotesize \captionsetup{font=footnotesize}
\begin{tabular}{c c c}
\toprule
Blockade & {\jap Genbutsu} & Safe\\
\midrule
{\Large\wan{2}} or {\Large\wan{3}} & {\Large\wan{7}} & {\Large\wan{4}}\\
{\Large\wan{3}} or {\Large\wan{4}} & {\Large\wan{8}} & {\Large\wan{5}}\\
{\Large\wan{4}} or {\Large\wan{5}} & {\Large\wan{9}} & {\Large\wan{6}}\\
{\Large\wan{5}} or {\Large\wan{6}} & {\Large\wan{1}} & {\Large\wan{4}}\\
{\Large\wan{6}} or {\Large\wan{7}} & {\Large\wan{2}} & {\Large\wan{5}}\\
{\Large\wan{7}} or {\Large\wan{8}} & {\Large\wan{3}} & {\Large\wan{6}}\\
\bottomrule
\end{tabular}}
\caption{Blockade and {\jap suji}} \label{tbl:kabesuji} \vsp
\end{floatingtable}

When we have a blockade of {\large\wan{2}} \emph{and} a {\large\wan{7}} is among a player's {\jap genbutsu}, we can deny not only a {\large\wan{1}-\wan{4}} {\jap suji} wait but also a {\large\wan{4}-\wan{7}} {\jap suji} wait, making {\large\wan{4}} safe. 
Combining the blockade and {\jap suji} theories like this might seem a bit complicated at first, but you will get used to it as you play more games. 

\bigskip
\subsection{Safety ranking}
Based on what we have learned so far, Table \ref{tbl:safety} on the next page provides a ranking of tile safety. 

\begin{table}[t!]\centering\small\captionsetup{font=small}
\caption{Safety ranking} \label{tbl:safety}
\begin{tabularx}{350pt}{X l}
\toprule
Rank & \\
\midrule
100\% & {\jap Genbutsu} \\
AAA & Fourth {\jap suji} terminal; Fourth honor tile\\
AA & Third {\jap suji} terminal; Third honor tile\\
AA- & Second {\jap suji} terminal\\
A+ & Second valueless Wind tile; First {\jap suji} terminal\\
A & Second honor tile\\
BBB & {\jap Suji} 4,5,6; No-chance tile\\
BB+ & {\jap Suji} 2, 8\\
BB- & {\jap Suji} 3, 7; One-chance tile (earlier turns)\\
B & First honor tile\\
CC & Non-{\jap suji} terminal\\
C & One-chance tile (later turns)\\
DDD & Non-{\jap suji} 2,8\\
DD & Non-{\jap suji} 3,7\\
D & Non-{\jap suji} 4,5,6\\
\bottomrule
\end{tabularx}
\end{table}

\bigskip
There is not much difference between ranks if they are given the same alphabet. 
Tiles in the AAA ranking are dangerous only against Thirteen Orphans. When Thirteen Orphans is not possible (i.e., there are some terminals or honors that are already exhausted), they become 100\% safe. Fourth tiles mean that three of that tile have already been discarded. Likewise, third, second, and first tiles mean that two, one, or none of that tile have already been discarded, respectively. 

\bigskip
There is a difference between 4,5,6 tiles, 3,7 tiles, 2,8 tiles, and terminals (1,9) because of the difference in versatility. Non-{\jap suji} 4,5,6 tiles are the most dangerous because they can be caught by two different {\jap suji} waits. For example, 4 can be caught by a 1-4 {\jap suji} and a 4-7 {\jap suji}, making it doubly dangerous. 3,7 tiles are more dangerous than 2,8 tiles because 3,7 can be caught by an edge wait, whereas 2,8 tiles cannot. terminals cannot be caught by either an edge wait or closed wait. {\jap Suji} 4,5,6 tiles are safer than {\jap suji} 2,8 tiles because 4,5,6 make for bad candidates for pair wait or single wait. 

\newpage
\section{Defense against {\jap riichi}} \label{sec:defense_riichi}

Putting together what we have learned so far, the defense strategy against an opponent's {\jap riichi} can be summarized as follows. 

\begin{itembox}[c]{Defense against {\jap riichi}}
\bi
\i Do not discard {\bf Rank D} tiles against an opponent's {\jap riichi} until your 
hand becomes ready\\(unless your hand has a really good wait \emph{and} a really high score).
\i If you need to push when your hand is 1-away from ready, you can
discard {\bf Rank C} or safer tiles. Only if your hand has a guaranteed {\jap mangan}, you can discard {\bf Rank D} tiles. 
\i If you need to push when your hand is 2-away from ready or worse,
you can discard {\bf Rank B} or safer tiles. 
\i If you cannot satisfy the above criteria, you must completely fold ({\jap betaori}). 
\ei
\end{itembox}

\subsection{What to discard when you get stuck}
When you cannot identify safe tiles at all, rely on the following and try to be as safe as possible. 

\subsubsection*{Tile chunks}
Discard pairs and concealed sets. Once you get one tile passed against a {\jap riichi}-ed player, you can be safe for the next turn or two. 

\subsubsection*{Avoid dealing into expensive hands}
If you discard terminals, you can avoid dealing into a {\jap tanyao} (All Simples) hand. 
Also, try not to discard the {\jap dora} indicator tile (when {\jap dora} is a number tile) and any tiles close to {\jap dora}, as well as the {\jap dora} tile itself.

\subsubsection*{Tiles outside early discards}
Tiles that are outside those discarded in ``early'' turns are relatively safe. 
Consider the following {\jap riichi}. 

\bp
\bei\wan{2}\fa\suo{9}\tong{3}\suo{7}\rwan{8}\\
\hspace{127pt}\footnotesize{{\jap riichi}}
\ep
This opponent discarded a {\large\wan{2}} in the 2nd turn, which is relatively early. This suggests that she is not very likely to have a {\large\wan{1}-\wan{4}} {\jap suji} wait. If she had a tile block {\large\wan{2}\wan{2}\wan{3}}, she would probably have kept it and discarded something else. 
This line of argument is obviously not 100\% reliable. However, if you compare {\large\wan{9}} and {\large\wan{1}} in the current example, {\large\wan{1}} is relatively safe. 

\subsection{Miscellaneous}
Here are some additional factors you may want to take into account when deciding whether or not to be defensive, and how much defensive you should be. 

\subsubsection*{Your position in the game}
You should be more defensive when you are ahead of the game, while you should be more aggressive when you are behind. This should especially be the case in the South round. 

\subsubsection*{Turn}
You can be more aggressive in earlier turns, whereas you need to be much more defensive towards the end of a hand. Suppose an opponent calls {\jap riichi} in the 3rd turn, and your hand is 1-away from ready. Since you have 15 more turns to draw, you still have a good chance of making the hand ready. In such situations, it may be worthwhile to be a little bit aggressive against {\jap riichi}. 
However, if you have only three more turns to draw (i.e., in the 15th turn) and your hand is still 1-away from ready, the chance of making a ready hand before the hand ends is very low. It is not worthwhile to discard dangerous tiles at this point. 

\bigskip
Moreover, in earlier turns, there are many {\jap suji} that are ``alive'', which diminishes the probability of dealing into an opponent's side-wait hands. For example, suppose an opponent calls {\jap riichi} with the following discard. 

\bp
\bei\fa\wan{9}\rsuo{2}\\
\hspace{70pt}\footnotesize{{\jap riichi}}
\ep
So far, only 2 out of 18 {\jap suji} have been denied by the discards ({\large\wan{6}-\wan{9}} and {\large\suo{2}-\suo{5}}), leaving 16 {\jap suji} alive. Suppose you are considering whether to discard a {\large\tong{1}}. Assuming that the {\jap riichi}-ed player has a side-wait hand, the conditional probability of dealing into her hand with a {\large\tong{1}} is only $\frac{1}{16}$ at this point.\footnote{Since the {\jap riichi}-ed player may not have a side-wait hand, the joint probability that this is a side-wait hand \emph{and} the hand waits for \tong{1} is even lower than $\frac{1}{16}$. The total probability that the {\jap riichi}-ed player is waiting for a \tong{1} is a bit greater than this joint probability because of the possibility of pair wait and single wait.}

\bigskip
However, as the hand proceeds, the number of live {\jap suji} waits will decrease, making it more and more dangerous to discard a non-{\jap suji} tile. Suppose that the hand proceeds and the {\jap riichi}-ed player's discard is as follows. 
\bp
\bei\fa\wan{9}\rsuo{2}\wan{3}\tong{5}\\
\vspace{-10pt}
\hspace{-5pt}\suo{4}\wan{4}\tong{6}\suo{6}\wan{2}\tong{7}\\
\vspace{-10pt}
\hspace{-115pt}\wan{8}
\ep
Since as many as 16 {\jap suji} waits have already been denied, there are only 2 {\jap suji} waits that are ``alive'' ({\large\tong{1}-\tong{4}} and {\large\suo{5}-\suo{8}}). Then, the conditional probability of dealing into her hand with a {\large\tong{1}} given that she has a side-wait hand is now as high as 50\%. This gives us an additional reason to be more defensive towards the end of a hand. 

\subsubsection*{Opponent's style}
If you know the type of opponent you are facing, you may want to take that into account. For example, if you know that your opponent is an old-fashioned player who calls {\jap riichi} only when they have a good-wait hand, you can rely heavily on {\jap suji} theories. 

\bigskip
However, if you know that your opponent understands the modern {\jap riichi} strategies as described in the previous chapter, it is more difficult for you to guess whether she has a good wait or a bad wait. This is because she would not shy away from {\jap riichi} even with a bad-wait hand. You cannot rely too much on {\jap suji} theories in such situations. 

\vfill

\section{Defense against melded hands} \label{sec:defense_meld}

\subsection{Guessing if an opponent has a ready hand}
To defend against a melded hand, we first need to know if an opponent has a ready hand or not. Again, we will use some simple shortcuts, which hopefully lead us to the right conclusion most (if not all) of the time. 

\bigskip
\begin{itembox}[c]{Defense against melded hands}
Assume an opponent has a ready hand in any of the following situations.
\be
\i She has three or more melded sets / runs.
\i When she is doing a flush hand, she starts discarding tiles in the suit she is supposedly collecting.
\i She keeps discarding the tile that she draws.
\ee
\vsps
\end{itembox}
\bigskip

\subsection{Estimating the value of an opponent's hand}

The next thing you need to know is how expensive an opponent's hand is. Although it is practically impossible to infer {\jap riichi}-ed player's hand value, we can often estimate the hand value of an opponent's melded hand. If you can easily see that an opponent's hand is {\jap tanyao}-only or {\jap fanpai}-only, there is no need to be defensive.

\subsubsection*{Melded hands with {\jap dora}}

An obvious case of an expensive melded hand is one with {\jap dora} tiles. 
For example, if an opponent has a melded set of {\jap dora}, clearly she has a four-fan (or higher) hand. Also, if you play with red fives, scores get expensive quite easily. For example, suppose the dealer has the following melded hand in East-1.
\vspace{-20pt}
\bp
\hspace{242pt}{\footnotesize\color{red!75!black} Red}\\ \vspace{-18pt}\suo{30}\suo{30}\suo{30}\suo{30}\suo{30}\suo{30}\suo{30}
\rdong\dong\dong~\rsuo{4}\rfs\suo{6}
\ep
Then, this hand has at least 5800 (seat and prevailing Wind + red five). Try not to push too hard against this player. 

\subsubsection*{Flush hands}

Flush hands ({\jap honitsu} or {\jap chinitsu}) tend to get expensive as well. 
\vspace{-20pt}
\bp
\hspace{145pt}{\footnotesize\color{red!75!black} Red}\\ \vspace{-18pt}\suo{30}\suo{30}\suo{30}\suo{30}
\rsuo{1}\suo{1}\suo{1}~\rsuo{4}\suo{3}\rfs~\rsuo{6}\suo{7}\suo{8}
\ep
A hand like above has a minimum of 3900 and a maximum of {\jap haneman} if you deal into it with a tile in Bamboos. You should fold when you draw an unwanted tile in Bamboos.

\subsubsection*{Value tiles}

Melded sets of value tiles also make for an expensive hand. 
\bp
\suo{30}\suo{30}\suo{30}\suo{30}
\rbai\bai\bai~\wan{9}\rwan{9}\wan{9}~\dong\dong\rdong~~\suo{1}\\
\hspace{330pt}\footnotesize{\jap Dora}
\ep
This is a pretty scary hand. You should not discard anything other than {\jap genbutsu} tiles. 

\bigskip
\subsubsection*{Cheap hands}

On the other hand, you can sometimes see that an opponent is likely to have a cheap hand. For example, suppose an opponent is doing the following. 

\bp
\suo{30}\suo{30}\suo{30}\suo{30}\suo{30}\suo{30}\suo{30}
\rsuo{7}\suo{6}\suo{8}~\rtong{5}\tong{6}\tong{7}
~~\suo{1}\\
\hspace{330pt}\footnotesize{\jap Dora}
\ep
We can see that {\jap honitsu} and {\jap ittsu} are impossible and {\jap sanshoku} is unlikely. If all of the {\jap dora} tiles and red fives are visible to you (in your hand or among the discards), you can be pretty sure that this hand is inexpensive. It is true that the following is still a possibility. 
\bp
\bai\bai\bai\fa\fa\fa\zhong~
\rsuo{7}\suo{6}\suo{8}~\rtong{5}\tong{6}\tong{7}
~~\suo{1}\\
\hspace{330pt}\footnotesize{\jap Dora}
\ep
But, you may be able to rule this out if you check what value tiles are still live in the board. 

\subsection{Identifying dangerous tiles against melded hands}

You can't win your own hand if you completely fold every time an opponent melds. Unless an opponent has an obviously expensive hand (e.g., three melded sets of value tiles, etc.), we would want not to fold completely but to discard some tiles that are not particularly dangerous. It is therefore important to identify dangerous tiles against melded hands. 

\bigskip
For example, consider the following. 
\bp
\suo{30}\suo{30}\suo{30}\suo{30}
\rfa\fa\fa~\wan{9}\wan{9}\rwan{9}~\tong{3}\rtong{3}\tong{3}
\ep
This opponent probably has a {\jap toitoi} (All Pungs) hand (otherwise, it would be cheap so you can ignore it). If it is {\jap toitoi}, {\jap suji} theories and blockade theories are completely useless (remember, they are useful against side-wait hands). 
Most dangerous tiles in this situation are ``raw'' tiles (tiles that are completely invisible to you). In particular, you should not discard raw value tiles. As single wait is also a possibility, all honor tiles are generally dangerous (unless they are the fourth tile). 

\bigskip
For another example, consider the following. 
\vspace{-10pt}
\bp
\suo{30}\suo{30}\suo{30}\suo{30}\suo{30}\suo{30}\suo{30}
\wan{9}\wan{9}\rwan{9}~\rsuo{6}\suo{7}\suo{8}~~\tong{6}\\
\hspace{325pt}\footnotesize{\jap Dora}
\ep
In a situation like this, one possibility is that the opponent has a pair-wait hand with value tiles on the one hand and {\jap dora} tiles on the other, as follows. 
\bp
\tong{6}\tong{6}\suo{1}\suo{2}\suo{3}\fa\fa
~\wan{9}\wan{9}\rwan{9}~\rsuo{6}\suo{7}\suo{8}~~\tong{6}\\
\hspace{325pt}\footnotesize{\jap Dora}
\ep
Especially when you are facing a ``reliable'' opponent, there would be a good reason (such as having two {\jap dora} tiles) why she rushed by melding a side-wait protorun first. 

\newpage
\subsection{Discard upon {\jap chii}}

We can sometimes identify tiles that are relatively safe or relatively dangerous against a melded hand by paying attention to what an opponent discarded upon calling the last {\jap chii} or {\jap pon}. Consider the following three cases. 

\subsubsection*{\fbox{Case 1} {\jap chii} $\rightarrow$ discard a tile in the same suit}
An opponent who had a melded set of {\large\fa} called {\jap chii} on a {\large\tong{3}} and discarded a {\large\tong{4}}. 
\begin{screen}
\bp
\suo{30}\suo{30}\suo{30}\suo{30}\suo{30}\suo{30}\suo{30}
\rtong{3}\tong{4}\tong{5}~\rfa\fa\fa\\{\normalsize and discarded a} \tong{4}
\ep
\end{screen}
In this case, it is unlikely that this opponent has a wait in the neighborhood of {\large\tong{4}}, so Dots tiles such as {\large\tong{2}\tong{3}\tong{5}\tong{6}\tong{7}} are relatively safe. 

\subsubsection*{\fbox{Case 2} {\jap chii} $\rightarrow$ discard a tile in a different suit}
An opponent who had a melded set of {\large\fa} called {\jap chii} on a {\large\tong{3}} and discarded a {\large\suo{4}}. 
\begin{screen}
\bp
\suo{30}\suo{30}\suo{30}\suo{30}\suo{30}\suo{30}\suo{30}
\rtong{3}\tong{4}\tong{5}~\rfa\fa\fa\\{\normalsize and discarded a} \suo{4}
\ep
\end{screen}
In this case, this opponent's wait is very likely to be in the neighborhood of the the last discard, {\large\suo{4}}. In particular, {\large\suo{2}-\suo{5}} {\jap suji} and {\large\suo{3}-\suo{6}} {\jap suji} are extremely dangerous, and a closed wait of {\large\suo{7}} is also a possibility. 

\subsubsection*{\fbox{Case 3} {\jap pon}}
An opponent who had a melded set of {\large\fa} called {\jap pon} on a {\large\tong{3}} and discarded a {\large\suo{4}}. 
\begin{screen}
\bp
\suo{30}\suo{30}\suo{30}\suo{30}\suo{30}\suo{30}\suo{30}
\rtong{3}\tong{3}\tong{3}~\rfa\fa\fa\\{\normalsize and discarded a} \suo{4}
\ep
\end{screen}
In this case, this opponent's wait can be in any suit; we cannot identify which tiles are particularly safe or dangerous. 

\bigskip
What makes these differences? 
These readings are based on an assumption that the opponent has a good 1-away  hand before calling the last {\jap chii} or {\jap pon}. In Case 1, the opponent has the following perfect 1-away hand before the last {\jap chii}. 
\bp
\wan{1}\wan{2}\wan{3}\wan{8}\wan{8}\tong{4}\tong{4}\tong{5}\suo{3}\suo{4}
~\rfa\fa\fa
\ep \index{1-away (1-{\jap shanten})!perfect 1-away}
Then, after calling {\jap chii} on a {\large\tong{3}} the opponent discards a {\large\tong{4}}, making the neighborhood of {\large\tong{4}} relatively safe. 

\bigskip
On the other hand, in Case 2, the opponent has the following perfect 1-away hand before the last {\jap chii}. 
\bp
\wan{1}\wan{2}\wan{3}\wan{8}\wan{8}\tong{4}\tong{5}\suo{3}\suo{4}\suo{4}
~\rfa\fa\fa
\ep
Then, after calling {\jap chii} on a {\large\tong{3}} the opponent discards a {\large\suo{4}}, making the neighborhood of {\large\suo{4}} dangerous. In this particular case, the resulting wait is {\large\suo{2}-\suo{5}} {\jap suji}. If the {\large\suo{3}\suo{4}\suo{4}} part were {\large\suo{4}\suo{4}\suo{5}}, the resulting wait is {\large\suo{3}-\suo{6}} {\jap suji}.  If the {\large\suo{3}\suo{4}\suo{4}} part were {\large\suo{4}\suo{6}\suo{8}}, the resulting wait is {\large\suo{7}}. 

\bigskip
Finally, in Case 3, the opponent has the following perfect 1-away hand before the last {\jap pon}. 
\bp
\wan{1}\wan{2}\wan{3}\wan{7}\wan{8}\tong{3}\tong{3}\suo{3}\suo{3}\suo{4}
~\rfa\fa\fa
\ep
Then, after calling {\jap pon} on a {\large\tong{3}} the opponent discards a {\large\suo{4}}, making the wait unrelated to {\large\suo{4}}. Notice that, if the opponent calls {\jap chii} on a {\large\wan{6}} with this hand and discards a {\large\suo{3}}, the neighborhood of the last discard becomes dangerous (just like Case 2). Similarly, if the opponent calls {\jap chii} on a {\large\suo{2}} with this hand and discards a {\large\suo{3}}, the neighborhood of the last discard becomes safe (just like Case 1). 

\bigskip
\begin{itembox}[c]{Discard upon {\jap chii}}
\bi
\i {\jap chii} $\rightarrow$ discard a tile in the same suit \\
	$\Rightarrow$ the neighborhood of the last discard is {\large safe}
\i {\jap chii} $\rightarrow$ discard a tile in a different suit \\
	$\Rightarrow$ the neighborhood of the last discard is {\large dangerous}
\i {\jap pon} \\
	$\Rightarrow$ wait can be anything
\ei
\vsps
\end{itembox}

\newpage
\section{Glossary}

\begin{description}
\item[{\jap Zentsu}] is to push to the fullest, usually against an opponent's {\jap riichi}. 
\item[{\jap Betaori}] is to fold to the fullest. 
\item[{\jap Suji}] is a three-tile interval that corresponds to the wait of a side-wait hand. There are six {\jap suji}: 1-4, 2-5, 3-6, 4-7, 5-8, and 6-9. See Section \ref{sec:suji}. 
	\index{suji@{\jap suji}}

\item[{\jap Suji} tile] is a tile that is made safe against side wait when a certain tile(s) is among an opponent's {\jap genbutsu}. For example, when a {\large\suo{4}} is among {\jap genbutsu}, {\large\suo{1}} and {\large\suo{7}} are safe against a side-wait hand. 

\item[{\jap Suji} trap] is when the wait is a {\jap suji} tile. When this happens, the wait is either pair wait, closed wait, edge wait, or single wait. 
\item[Blockade ({\jap kabe}; wall)] is formed when three or four of a number tile are visible to you. When all four of a number tile are visible, they form a complete blockade, making for no-chance tiles. When three of a number tile are visible, they form an incomplete blockade, making for one-chance tiles. 

\item[No-chance tile] is a tile that is made safe by a complete blockade. There is ``no chance'' that an opponent has a protorun that includes a tile that is blocked. 

\item[One-chance tile] is a tile that is made safe by an incomplete blockade. There is ``one chance'' that an opponent has a protorun that includes a tile that is blocked. 

\end{description}

