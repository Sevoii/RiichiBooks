%~~~~~~~~~~~~~~~~~~~~~~~~~~~~~~~~~~~~~~~~~~~~~~~~~
% {\jap riichi} Book 1, Chapter 7: {\jap riichi}
%~~~~~~~~~~~~~~~~~~~~~~~~~~~~~~~~~~~~~~~~~~~~~~~~~

\chapter{{\jap Riichi} judgement} \label{ch:riichi}

\section{To {\jap riichi} or not to {\jap riichi}?}
{\jap Riichi} is a really powerful tool in Riichi Mahjong. Once you {\jap riichi}, the opponents would have to slow down their attacks or even completely fold to avoid dealing into your hand. Therefore, one of our top priorities in playing Riichi Mahjong is to try to make the hand ready as fast as possible and call {\jap riichi} before anyone else does.

\bigskip
At the same time, however, there are situations where you should keep {\jap dama} (i.e., not calling {\jap riichi} when having a closed ready hand). If you have played mahjong long enough, you must have come across many instances where you wondered if you should call {\jap riichi} or keep {\jap dama}. 
Knowing when to call {\jap riichi} is one of the most fundamental elements of mahjong strategies, yet it appears this is not very well understood among European players. 
Let's first review the pros and cons of calling {\jap riichi}. 

\begin{itembox}[c]{Demerits of {\jap riichi}}
	\bi \itemsep.1pt
	\i You have to pay 1000 points as a {\jap riichi} bet.
	\i The opponents may play defense and may not discard your winning tiles that they would otherwise discard.
	\i You cannot exchange tiles any more; you cannot play defense nor improve the wait / scores.
	\ei
\end{itembox}

\begin{itembox}[c]{Merits of {\jap riichi}}
	\bi
	\i You get one {\jap fan}. You may further get {\jap ippatsu} and {\jap ura dora}.
	\i The opponents may completely fold or play more defensively than otherwise. As a result, the opponents may fail to make their hand ready, in which case you will have more opportunities to draw tiles. 
	\i When your previous discards make it look that your winning tile is safe, {\jap riichi} may actually \emph{increase} the chance that the opponents discard your winning tile. 
	\ei
\end{itembox}

\bigskip
\noindent
Comparing these pros and cons, it should be evident that calling {\jap riichi} really is a low-risk \& high-return offense tactic. Moreover, {\jap riichi} could also work as a defense tactics. If you {\jap riichi} before the opponents do, it may prevent them from building a ready hand, which obviously reduces your chance of dealing into their hand. 
Although calling {\jap riichi} means that you can no longer play defensively by choosing safe tiles to discard, it poses less of a problem if your defensive skills are not very good. 

\bigskip

{\jap Riichi} judgement criteria I recommend are summarized as follows. 

\bigskip
\begin{itembox}[c]{{\jap Riichi} judgement}
Choose {\jap riichi} over {\jap dama} if \underline{\large at least one} of the following three conditions is met. 
\bi
\i[] 1. Your hand has at least one {\jap fan} other than {\jap riichi}.
\i[] 2. Your hand has a good wait.
\i[] 3. You are the dealer.
\ei \vsps
\end{itembox}

\newpage
\noindent This means that you should call {\jap riichi} if 
\bi \itemsep.1em
\i your hand has a bad wait but has one {\jap fan} or more (including {\jap dora}) other than {\jap riichi};
\i your hand is {\jap riichi}-only but with a good wait, or;
\i your hand is {\jap riichi}-only with a bad wait, but you are the dealer. 
\ei
The only type of {\jap riichi} that these criteria prohibit is a bad-wait {\jap riichi}-only hand by a non-dealer. 


\section{Insta-{\jap riichi}} 	\index{insta-{\jap riichi}}
Keep in mind that, when you call {\jap riichi} you should do so \emph{immediately} when your hand becomes ready ({\bf insta-{\jap riichi}}). There is usually no point in waiting for a few turns to have the ``right'' moment to {\jap riichi}. 

\bigskip
Let me describe a few frequently seen examples where you should do insta-{\jap riichi}. 
In all the examples that follow, we assume that you are the South player in the 6th turn in East-1. 

\subsection{Examples of insta-{\jap riichi}}

\begin{itembox}[r]{{\jap Pinfu}-only hand}
\bp
\wan{7}\wan{8}\wan{9}\tong{3}\tong{4}\tong{5}\tong{8}\tong{8}\suo{2}\suo{3}\suo{3}\suo{6}\suo{7}\suo{8}~~\bei\\
\hspace{311pt}\footnotesize{\jap Dora}
\ep \index{pinfu@{\jap pinfu}}
\vspace{-15pt}
\end{itembox}

\noindent You should do insta-{\jap riichi} with a {\jap pinfu}-only hand. It is true that this hand will have {\jap tanyao} if you draw a {\large \wan{6}} and discard the {\large \wan{9}}, but waiting for that to happen is simply inefficient. Even after you replace the {\large\wan{9}} with a {\large\wan{6}}, you will lose {\jap tanyao} anyway if you win the hand on a {\large\suo{1}}. Getting either {\jap ippatsu} or one {\jap ura dora} has a much higher probability than drawing a {\large \wan{6}} first and then winning on a {\large \suo{4}}. 

\begin{center}
{\large $\Rightarrow$ Insta-{\jap riichi}, discarding a \suo{3}~!}
\end{center}

\bigskip
\begin{itembox}[r]{Bad wait, one {\jap dora}}
\bp
\wan{3}\wan{4}\wan{5}\wan{6}\wan{8}\tong{7}\tong{8}\tong{9}\suo{1}\suo{1}\bei\bei\bei\xi~~\tong{9}\\
\hspace{311pt}\footnotesize{\jap Dora}
\ep
\vspace{-15pt}
\end{itembox}

\noindent Since this hand has one {\jap dora}, you should do insta-{\jap riichi}. Do not shy away from {\jap riichi} even with closed-wait or edge-wait hands. It is true that the wait can be improved with as many as four kinds of tiles ({\large \wan{2}\wan{4}\wan{5}\suo{1}}), but drawing one of those would take about eight more turns, on average.\footnote{This rough calculation is based on an assumption that the probability of drawing an arbitrary tile is $\frac{1}{34}=$ about 3\%.} 
Since this is a {\jap yaku}-less hand, you cannot win it by {\jap ron} while waiting in {\jap dama}.  
Moreover, even when the wait gets improved, this hand will never become {\jap pinfu} anyway, so the score will not be improved. 

\begin{center}
{\large $\Rightarrow$ Insta-{\jap riichi}, discarding the \xi ~!}
\end{center}

\begin{itembox}[r]{1-away from {\jap sanshoku}}
\bp
\hspace{-230pt}{\footnotesize\color{red!75!black} Red}\\ \vspace{-18pt}
\wan{3}\wan{4}\rfw\wan{6}\wan{8}\tong{6}\tong{7}\tong{8}\suo{7}\suo{8}\suo{9}\bei\bei\dong~~\tong{9}\\
\hspace{311pt}\footnotesize{\jap Dora}
\ep \index{sanshoku@{\jap sanshoku}}
\vspace{-15pt}
\end{itembox}

\noindent It is true that there are some tiles that can improve the scores and/or the wait of this hand. For example, if you draw a {\large\suo{6}}, the hand will have {\jap sanshoku} (Mixed Triple Chow). If you draw any of {\large\wan{2}\wan{4}\wan{5}}, the hand will have {\jap pinfu}. However, since the hand already has one {\jap fan} (red five), you should do insta-{\jap riichi}. 

\begin{center}
{\large $\Rightarrow$ Insta-{\jap riichi}, discarding the \dong ~!}
\end{center}

\bigskip
\begin{itembox}[r]{Unconfirmed {\jap sanshoku}}
\bp
\wan{3}\wan{4}\wan{9}\wan{9}\tong{1}\tong{2}\tong{3}\tong{4}\tong{5}\tong{6}\tong{7}\suo{2}\suo{3}\suo{4}~~\bei\\
\hspace{311pt}\footnotesize{\jap Dora}
\ep \index{sanshoku@{\jap sanshoku}}
\vspace{-15pt}
\end{itembox}

\noindent You would want to win this hand on a {\large\wan{2}} (rather than on a {\large\wan{5}}) so that you can claim {\jap sanshoku}. However, waiting for a {\large\wan{2}} without {\jap riichi} is absurd. The worst case scenario is to draw a {\large\wan{5}} while waiting in {\jap dama}, in which case you only get 400-700. If you {\jap riichi} and draw a {\large\wan{5}}, you will get at least 700-1300. With one {\jap ura dora} or {\jap ippatsu} you will get 1300-2600. 
\index{sanshoku@{\jap sanshoku}}
\begin{center}
{\large $\Rightarrow$ Insta-{\jap riichi}, discarding the \tong{1}~!}
\end{center}

\bigskip
\begin{itembox}[r]{{\jap Riichi}-only hand}
\bp
\wan{7}\wan{8}\wan{9}\tong{3}\tong{4}\tong{5}\suo{2}\suo{3}\suo{6}\suo{7}\suo{8}\suo{9}\bai\bai~~\bei\\
\hspace{311pt}\footnotesize{\jap Dora}
\ep
\vspace{-15pt}
\end{itembox}

\noindent This hand has no {\jap yaku} or {\jap dora}, but the wait is good. You can do insta-{\jap riichi} with a {\jap riichi}-only hand as long as the hand has a good wait. 

\begin{center}
{\large $\Rightarrow$ Insta-{\jap riichi}, discarding the \suo{9}~!}
\end{center}


\bigskip
\begin{itembox}[r]{Waiting for {\jap dora}}
\bp
\wan{2}\wan{4}\tong{5}\tong{6}\tong{7}\tong{8}\tong{8}\tong{8}\suo{2}\suo{3}\suo{4}\suo{6}\suo{7}\suo{8}~~\wan{3}\\
\hspace{311pt}\footnotesize{\jap Dora}
\ep
\vspace{-15pt}
\end{itembox}

\noindent You should do insta-{\jap riichi} even when waiting for {\jap dora}. 

\begin{center}
{\large $\Rightarrow$ Insta-{\jap riichi}, discarding a \tong{8}~!}
\end{center}

\noindent When you wonder whether or not you should {\jap riichi} in a given situation, choose {\jap riichi}. You will be correct most of the time. 

\subsection{Good wait vs. high scores}
\bigskip
We have discussed in previous chapters the difficult tradeoff we face between speed (tile efficiency) and high scores. In {\jap riichi} judgement, this tradeoff manifests itself as a choice between (a) having a good wait with lower scores and (b) pursuing higher scores with a bad wait.

\bigskip
In the following examples, there are more than one discard candidates to make the hand ready. I will discuss how to take a balance of tile efficiency and hand value in calling {\jap riichi}. Again, we will assume that you are the South player in the 6th turn in East-1.

\newpage
\begin{itembox}[r]{Cheap hand}
\bp
\wan{7}\wan{8}\wan{9}\tong{3}\tong{4}\tong{5}\suo{2}\suo{3}\suo{3}\suo{6}\suo{7}\suo{8}\bai\bai~~\bei\\
\hspace{311pt}\footnotesize{\jap Dora}
\ep
\vspace{-15pt}
\end{itembox}

\noindent Calling {\jap riichi} by discarding a {\large\suo{3}} makes for a good wait ({\large \suo{1}-\suo{4}}; 2 kinds--8 tiles), but the hand becomes {\jap riichi}-only. On the other hand, calling {\jap riichi} by discarding the {\large\suo{2}} leaves the possibility that the hand has an additional {\jap yaku} (White Dragon), although the wait is not as good ({\large \suo{3} \bai}; 2 kinds--4 tiles). Which one should we choose?

\bigskip
In cases like this, you should aim for high scores by discarding the {\large\suo{2}}. 
Discarding the {\large\suo{2}} and winning on a {\large\bai} would double the hand value ({\jap riichi} + White Dragon = 2600 versus {\jap riichi}-only = 1300), but the probability of winning on a {\large\bai} would not shrink below half of the probability of winning on {\large\suo{1}-\suo{4}}. 

\begin{center}
{\large $\Rightarrow$ Insta-{\jap riichi}, discarding the \suo{2}~!}
\end{center}

\noindent Here is a simple rule of thumb: when the \emph{minimum} (guaranteed) hand value is below 5200 (when won by {\jap ron}), you should value scores over wait. When the \emph{minimum} hand value is 5200 or above, you should value wait over scores. 
We use 5200 as a cut-point because an additional {\jap fan} (roughly) doubles the hand value until it reaches 5200. 

\bigskip
\begin{itembox}[r]{More expensive hand}
\bp
\wan{7}\wan{8}\wan{9}\tong{3}\tong{4}\tong{5}\suo{2}\suo{3}\suo{3}\suo{6}\suo{7}\suo{8}\bai\bai~~\wan{8}\\
\hspace{311pt}\footnotesize{\jap Dora}
\ep
\vspace{-15pt}
\end{itembox}

\noindent This hand has one {\jap dora}, but the minimum hand value is still below 5200 ({\jap riichi} + {\jap dora} = 2600). Therefore, again, you should value scores over wait and do insta-{\jap riichi} by discarding the {\large \suo{2}}. 


\bigskip
\begin{itembox}[r]{Expensive hand}
\bp
\wan{7}\wan{8}\wan{9}\tong{3}\tong{4}\tong{5}\suo{4}\suo{5}\suo{5}\bai\bai\bai\fa\fa~~\wan{9}\\
\hspace{311pt}\footnotesize{\jap Dora}
\ep
\vspace{-15pt}
\end{itembox}

\noindent With this hand, calling {\jap riichi} guarantees 5200 ({\jap riichi} + White Dragon + one {\jap dora}). Therefore, you should value wait over scores this time. It goes without saying that {\jap riichi} is better than going {\jap dama}. 

\begin{center}
{\large $\Rightarrow$ Insta-{\jap riichi}, discarding a \suo{5}~!}
\end{center}

\bigskip
\begin{itembox}[r]{Good wait vs. {\jap sanshoku}}
\bp
\wan{2}\wan{4}\wan{5}\suo{2}\suo{3}\suo{4}\suo{7}\suo{7}\tong{2}\tong{3}\tong{4}\tong{9}\tong{9}\tong{9}~~\bei\\
\hspace{311pt}\footnotesize{\jap Dora}
\ep \index{sanshoku@{\jap sanshoku}}
\vspace{-15pt}
\end{itembox}

\noindent Calling {\jap riichi} by discarding the {\large \wan{2}} gives you {\jap riichi}-only with a good wait, whereas calling {\jap riichi} by discarding the {\large \wan{5}} gives you {\jap riichi} + {\jap sanshoku} with a bad wait. Since {\jap riichi}-only is short of 5200, you should value scores over wait. 

\begin{center}
{\large $\Rightarrow$ Insta-{\jap riichi}, discarding the \wan{5}~!}
\end{center}

\bigskip
\begin{itembox}[r]{Good wait vs. {\jap ittsu}}
\bp
\wan{1}\wan{3}\wan{4}\wan{5}\wan{5}\wan{6}\wan{7}\wan{8}\wan{9}\tong{3}\tong{4}\tong{5}\tong{9}\tong{9}~~\bei\\
\hspace{311pt}\footnotesize{\jap Dora}
\ep \index{ittsu@{\jap ittsu}}
\vspace{-15pt}
\end{itembox}

\noindent Calling {\jap riichi} by discarding the {\large \wan{1}} gives you only {\jap riichi} + {\jap pinfu} with a good wait, whereas calling {\jap riichi} by discarding a {\large \wan{5}} gives you {\jap riichi} + {\jap ittsu} with a bad wait. Since {\jap riichi} + {\jap pinfu} is short of 5200, you should value scores over wait. 
\begin{center}
{\large $\Rightarrow$ Insta-{\jap riichi}, discarding a \wan{5}~!}
\end{center}


\bigskip
\begin{itembox}[r]{Expensive {\jap pinfu} hand}
\bp
\hspace{-17pt}{\footnotesize\color{red!75!black} Red}\hspace{155pt}{\footnotesize\color{red!75!black} Red}
\\ \vspace{-18pt}
\wan{1}\wan{3}\wan{4}\rfw\wan{6}\wan{7}\wan{7}\wan{8}\wan{9}\tong{3}\tong{4}\rfd\tong{9}\tong{9}~~\bei\\
\hspace{311pt}\footnotesize{\jap Dora}
\ep
\end{itembox}

\noindent Discarding a {\large\wan{7}} would give you a {\jap dama mangan} hand with a bad wait. That is not too bad, but it is much better to {\jap riichi} by discarding the {\large\wan{1}} ({\jap riichi} + {\jap pinfu} + two red fives = 7700 with a very good wait). 

\begin{center}
{\large $\Rightarrow$ Insta-{\jap riichi}, discarding the \wan{1}~!}
\end{center}

\bigskip

\noindent Let's summarize what we have learned. 

\bigskip

\begin{itembox}[c]{Good wait or high scores?}
\bi
\i Scores are more important than wait\\when the minimum hand value is < 5200.
\i Wait is more important than scores\\when the minimum hand value is $\geq$ 5200.
\ei
\end{itembox}

\newpage
\subsection{{\jap Chiitoitsu} waiting for {\jap dora}}
	\index{chiitoitsu@{\jap chiitoitsu} (Seven Pairs)}
We will discuss {\jap riichi} criteria for a {\jap chiitoitsu} (Seven Pairs) hand in this and the next sections. 

\bigskip
\begin{itembox}[r]{{\jap Chiitoitsu 1}}
\bp
\wan{2}\wan{2}\tong{1}\tong{1}\tong{4}\tong{9}\tong{9}\suo{2}\suo{2}\suo{4}\zhong\zhong\fa\fa~~\tong{4}\\
\hspace{310pt}\footnotesize{\jap Dora}
\ep
\vspace{-15pt}
\end{itembox}

\noindent You should not shy away from {\jap riichi} with a {\jap chiitoitsu} hand, especially when waiting for {\jap dora}. Since {\jap dora} of {\large \tong{4}} is something your opponents wouldn't lightly discard even if you keep {\jap dama}, the chance of winning this hand by {\jap ron} is not very high anyway. Therefore, you would rather {\jap riichi} and aim to improve the score. 

\bigskip
A two-{\jap dora} {\jap chiitoitsu} hand can be a game-deciding hand that secures you the first place in a game. If you {\jap tsumo} after {\jap riichi}, it is at least {\jap haneman} and it can easily be {\jap baiman} with {\jap ura dora} ({\jap ura dora} always come in pairs with a {\jap chiitoitsu} hand). Even when you win by {\jap ron}, it will be {\jap haneman} with either {\jap ura dora} or {\jap ippatsu}. 

\begin{center}
{\large $\Rightarrow$ Insta-{\jap riichi}, discarding the \suo{4}~!}
\end{center}

\bigskip
\begin{itembox}[r]{{\jap Chiitoitsu 2}}
\bp
\wan{2}\wan{2}\tong{1}\tong{1}\tong{9}\tong{9}\xi\suo{2}\suo{2}\suo{4}\zhong\zhong\fa\fa~~\xi\\
\hspace{310pt}\footnotesize{\jap Dora}
\ep
\vspace{-15pt}
\end{itembox}
{\jap Dora} in this example is a value-less Wind tile, which may be easily discarded by your opponents if you keep {\jap dama}. Nevertheless, you should still do insta-{\jap riichi} by discarding the {\large \suo{4}}. Aim for {\jap haneman} or {\jap baiman} rather than being content with 6400. 

\begin{center}
{\large $\Rightarrow$ Insta-{\jap riichi}, discarding the \suo{4}~!}
\end{center}

\bigskip

\noindent {\jap Riichi} criteria for {\jap chiitoitsu} hands waiting for {\jap dora} are really simple. 

\begin{itembox}[c]{{\jap Riichi} judgement for {\jap chiitoitsu}}
{\jap Riichi} any {\jap chiitoitsu} hand if waiting for {\jap dora}!
\end{itembox}

\bigskip

\subsection{{\jap Chiitoitsu} not waiting for {\jap dora}}
{\jap Riichi} criteria for {\jap chiitoitsu} hands get a bit more complicated when you are not waiting for {\jap dora}, summarized as follows.

\begin{itembox}[c]{{\jap Riichi} judgement for {\jap chiitoitsu}}
Do insta-{\jap riichi} with {\jap chiitoitsu} (not waiting for {\jap dora}) if one or more of the following holds:

\bi\itemsep.1em
\i You are the dealer;
\i You have {\jap tanyao};
\i You have one red five, waiting for a regular five;
\i The wait is a {\jap suji}-trap wait;\footnote{An example of a {\jap suji}-trap wait is: you are waiting for a 3, and a 6 in the same suit is among your discards. See Chapter \ref{ch:defense}.} \index{suji@{\jap suji}!{\jap suji} trap}
\i The wait is any tile other than 4, 5, 6;
\i The score without {\jap riichi} is below {\jap mangan}.
\ei \vsps
\end{itembox}

\bigskip
\noindent This means that the only two cases where you should go {\jap dama} are (a) when you are a non-dealer \emph{and} you are waiting for 4,5,6, and (b) you have {\jap honitsu} (Half Flush) or {\jap chinitsu} (Full Flush) {\jap chiitoitsu}.\footnote{It is also (theoretically) possible to have  {\jap tsu iso} (All Honors) {\jap chiitoitsu}. Do whatever you want with such a once-in-lifetime hand. I would {\jap riichi}.} The criteria do not change when your hand already has two {\jap dora} and waits for a non-{\jap dora} tile.

\bigskip
\begin{itembox}[r]{{\jap Chiitoitsu 3}}
\bp
\wan{3}\wan{4}\wan{4}\tong{1}\tong{1}\tong{6}\tong{6}\suo{2}\suo{2}\suo{2}\xi\xi\fa\fa~~\bei\\
\hspace{310pt}\footnotesize{\jap Dora}
\ep
\vspace{-15pt}
\end{itembox}

\noindent If you make the hand ready for {\jap chiitoitsu}, you will be waiting for a {\large\wan{3}}, a non-4,5,6 tile. {\jap riichi} is better than {\jap dama} in this case. 

\begin{center}
{\large $\Rightarrow$ Insta-{\jap riichi}, discarding a \suo{2}~!}
\end{center}

\bigskip
\begin{itembox}[r]{{\jap Chiitoitsu 4}}
\bp
\wan{4}\wan{4}\wan{6}\tong{6}\tong{6}\tong{8}\tong{8}\suo{2}\suo{2}\suo{2}\suo{5}\suo{5}\suo{7}\suo{7}~~\bei\\
\hspace{310pt}\footnotesize{\jap Dora}
\ep
\vspace{-15pt}
\end{itembox}

\noindent Waiting for a 6 is not ideal, but having {\jap tanyao} justifies {\jap riichi}. Aim for 6400 {\jap ron} or {\jap mangan} {\jap tsumo}.

\begin{center}
{\large $\Rightarrow$ Insta-{\jap riichi}, discarding a \suo{2}~!}
\end{center}

\bigskip
\begin{itembox}[r]{{\jap Chiitoitsu 5}}
\bp
\hspace{-100pt}{\footnotesize\color{red!75!black} Red}\\ \vspace{-18pt}
\wan{2}\wan{2}\tong{1}\tong{1}\tong{4}\rfd\tong{9}\tong{9}\suo{2}\suo{2}\zhong\zhong\fa\fa~~\bei\\
\hspace{310pt}\footnotesize{\jap Dora}
\ep
\vspace{-15pt}
\end{itembox}

\noindent Likewise, waiting for a 5 is not ideal, but having a red five justifies {\jap riichi}. 

\begin{center}
{\large $\Rightarrow$ Insta-{\jap riichi}, discarding the \tong{4}~!}
\end{center}

\bigskip
\begin{itembox}[r]{{\jap Chiitoitsu 6}}
\bp
\suo{1}\suo{1}\suo{2}\suo{2}\suo{4}\suo{4}\suo{7}\suo{8}\suo{8}\suo{9}\fa\fa\bai\bai~~\bei\\
\hspace{310pt}\footnotesize{\jap Dora}
\ep
\vspace{-15pt}
\end{itembox}

\noindent Since you can get {\jap mangan ron} or {\jap haneman tsumo} without {\jap riichi}, you should keep {\jap dama} with this hand (discard the {\large \suo{7}}). 

\bigskip
\begin{itembox}[r]{{\jap Chiitoitsu 7}}
\bp
\wan{4}\wan{4}\wan{6}\tong{6}\tong{6}\tong{8}\tong{8}\suo{2}\suo{2}\suo{4}\suo{5}\suo{5}\fa\fa~~\bei\\
\hspace{310pt}\footnotesize{\jap Dora}
\ep
\vspace{-15pt}
\end{itembox}

\noindent With this hand, you should keep {\jap dama} unless you are the dealer. However, if you have already discarded a {\large\wan{3}} \& a {\large\wan{9}} or a {\large\suo{1}} \& a {\large\suo{7}}, making for a {\jap suji}-trap wait, you can call {\jap riichi}. You can also call {\jap riichi} if your wait is ``cheap'' in the board (more on this in the next section).\index{suji@{\jap suji}!{\jap suji} trap}

\bigskip
Why should we refrain from calling {\jap riichi} with a single wait of 4, 5, 6?
Because of the differences in versatility,\footnote{Recall our discussions of tile versatility on in Section \ref{sec:versatility}.} some tiles make for a better single wait than others. 
Specifically, valueless Wind tiles are the best candidate for a single wait, followed by value tiles, terminals (1s and 9s), and then simple tiles. Among simple tiles, 2s and 8s are better than 3s and 7s, and 4,5,6 tiles make for the worst kind of single wait. 

\bigskip
Since having 4,5,6 tiles is crucial to utilize red fives, your opponents are not very likely to discard them. Moreover, single waits of 4, 5, 6 are less likely to become a {\jap suji}-trap wait before or after calling {\jap riichi}, compared with single waits of 1,2,3,7,8,9. \index{suji@{\jap suji}!{\jap suji} trap}
A single wait of 4 requires both 1 and 7 to be discarded to become a {\jap suji}-trap wait. A single wait of 1, on the other hand, only requires 4 to be discarded. 

\newpage
\section{When \emph{not} to {\jap riichi}} \label{sec:dama} \index{dama@{\jap dama}}
Keeping a hand {\jap dama} for no reason is one of the two biggest sins in Riichi Mahjong (the other will be introduced in Chapter \ref{ch:call}). {\bf Do not ever do meaningless {\jap dama}.} To put it the other way around, it is OK to keep {\jap dama} if there is a meaning to do so. 

\bigskip
That said, there are not many instances where going {\jap dama} is better than calling {\jap riichi}. It is thus useful to memorize all these exceptional cases; then you should call {\jap riichi} in all other instances. Here is a list of five situations where {\jap dama} is justifiable. 

\bigskip
\begin{itembox}[c]{Reasons to keep {\jap dama}}
\be \itemsep.2em
\i[]\ref{sec:badwaits} Bad wait and no {\jap dora}
\i[]\ref{sec:lead} In the lead
\i[]\ref{sec:genbutsu} {\jap Genbutsu} wait
\i[]\ref{sec:high} Expensive hand
\i[]\ref{sec:imp} Many possibilities of improving the hand
\ee
\end{itembox}

\bigskip
\subsection{Bad wait} \label{sec:badwaits}
	\index{waits}
It is OK to go {\jap dama} if the wait of your hand is really bad, especially when your hand has at least one {\jap yaku} without {\jap riichi} so you can win it by {\jap ron}. The question then is, what is a really bad wait? The answer depends on three things:
\be \itemsep.2em
\i the kinds and the number of winning tiles left;
\i whether your wait is ``expensive'' according to your reading of the board; and
\i whether your wait is likely to appear safe in the eyes of your opponents.
\ee

\bigskip
\subsubsection{1. The number of winning tiles left}
\bigskip
The first factor to consider is the pure number of winning tiles of your hand. The more tiles you can win on, the better the wait is. Table \ref{tbl:repwaits} provides a list of representative waits (roughly) in the order of desirability. 

	\index{waits!side wait} \index{waits!closed wait} 
	\index{waits!pair wait} \index{waits!edge wait} \index{waits!single wait} 
	\index{waits!stretched single wait} \index{waits!semi side wait} 
{\begin{table}[h!]\centering\captionsetup{font=small}
\caption{Typical wait patterns} \label{tbl:repwaits}
\begin{tabular}{l c c c}
\toprule
Name & Example & Wait & Kinds \& Number\\
\midrule
side wait & {\large\wan{3}\wan{4}} & {\large\suo{2}-\suo{5}} & 2 kinds--8 tiles\\[\sep]
semi side wait & {\large\tong{2}\tong{2}\tong{3}\tong{4}} & {\large \tong{2}-\tong{5}} & 2 kinds--6 tiles\\[\sep]
stretched single & {\large \suo{2}\suo{3}\suo{4}\suo{5}} & {\large \suo{2}-\suo{5}} & 2 kinds--6 tiles\\[\sep]
pair wait & {\large \wan{2}\wan{2}\tong{4}\tong{4}}& {\large \wan{2}\tong{4}} & 2 kinds--4 tiles\\[\sep]
closed wait & {\large \suo{2}\suo{4}} & {\large \suo{3}} & 1 kind--4 tiles\\[\sep]
edge wait & {\large \wan{1}\wan{2}} & {\large \wan{3}} & 1 kind--4 tiles\\[\sep]
single wait & {\large \tong{2}} & {\large \tong{2}} & 1 kind--3 tiles\\[\sep]
\bottomrule
\end{tabular}
\vspace{-10pt}
\end{table}}

\bigskip
In general, wait is said to be good if there are at least two kinds and more than four tiles left to win on. Therefore, pair wait, closed wait, edge wait, and single wait are generally considered to be a bad wait. 

\bigskip
In counting the kinds and the number of winning tiles for your hand, keep in mind that you have to count the kinds and the number of \emph{live} tiles to win on.
For example, if your opponents have already discarded all the four tiles of {\large \suo{2}} somehow, a side wait of {\large \suo{2}-\suo{5}} essentially becomes an edge wait of {\large\suo{5}}, leaving only 1 kind--4 tiles to win on.

\subsubsection{2. Cheap / expensive waits}
However, this is only a part of the picture. When judging whether your wait is good enough, you should also take into account the second factor; namely, whether or not your winning tiles are likely to be used by other players in their hands. As we learned in discussing {\jap chiitoitsu} hands, middle simple tiles 4, 5, and 6 generally have a high chance of being used by the opponents. 

\bigskip
Judging whether certain tiles are likely to be used by the opponents also involves a bit of board reading. If your opponents have already discarded a lot of tiles in Bamboos, for example, we say that Bamboos are ``cheap'' in the board. Cheap waits are good waits. Suppose three or more of {\large\suo{4}} have already been discarded by your opponents. In such situations, an edge wait of {\large \suo{3}} is not bad at all. This is because the paucity of {\large\suo{4}} makes it rather difficult for anyone to utilize a {\large\suo{3}} in their hand. There is also a good chance that {\large\suo{3}} tiles are live in the wall. Even when an opponent draws a {\large\suo{3}} after you {\jap riichi}, they will have difficulty utilizing it in their hand; they have to either discard the {\large\suo{3}} or completely fold. 

\bigskip
Applying the same logic, we can see why honor tiles make for a good wait. 
For example, suppose a {\large\zhong} had already been discarded and you now have a pair of {\large\zhong} in your hand. Then, the opponents will have difficulty utilizing the last {\large\zhong} when they draw it. In a situation like this, a pair wait of {\large\zhong} (and another tile) is almost as good as side wait, despite the fact that the number of tiles to win on is smaller. 

\bigskip
On the other hand, when tiles in one suit are not being discarded as much as those in the other two suits, that suit is being ``expensive'' in the board. For example, suppose one or more opponents are pursuing a flush hand (i.e., {\jap honitsu} / {\jap chinitsu}) in Bamboos. Then, even a side wait of {\large \suo{2}-\suo{5}} can be bad. 

\subsubsection{3. Trap waits}

The third factor you may want to consider is whether your wait would appear safe in the eyes of your opponents. For example, when you have a closed wait of {\large\suo{2}} and you have already discarded a {\large\suo{5}}, there is a good chance that your opponent is tricked into thinking that {\large\suo{2}} is safe. This is called a {\jap suji}-trap wait (see Chapter \ref{ch:defense}). \index{suji@{\jap suji}!{\jap suji} trap}
For another example, suppose someone has a concealed quad of {\large\wan{8}} when you happen to have a pair wait of {\large\wan{9}} and something else. Then, the opponents may think that a {\large\wan{9}} may be safe to discard even when it is not. 

\bigskip
That being said, reading the board requires some experience, and reading the opponents' thought is even more difficult. You may want to concentrate more on advancing your own hand rather than spending too much time trying to read the board. Just keep in mind that having a pair/closed/edge/single wait does not automatically mean that the wait is bad. Here is a rule of thumb to simplify your decision making.

\bigskip
\begin{itembox}[c]{A reason to keep {\jap dama}: Bad wait}
\bi
\i Call {\jap riichi} if there are three or more winning tiles left in the board.
\i Go {\jap dama} if there are only one or two simple tiles left to win on.\footnote{When waiting for an honor/terminal tile, you can call {\jap riichi} even when only one tile is left in the board.}
\ei \vsps
\end{itembox}

\subsection{In the lead}\label{sec:lead}
The second case where going {\jap dama} may be preferred to {\jap riichi} is when you are ahead of the game by much, and you just want to proceed to the next hand or finish the game while keeping your leading position. This is especially the case towards the end of the South round. For example, let's say you are in South-4 and the scores are as follows:

\begin{center}
\begin{tabular}{l r l r}
East (you) & 39000 & South & 22900\\
West & 13000 & North & 25100\\
\end{tabular}
\end{center}

You are currently in a safe lead because even the second ranked player (North) cannot defeat you with a {\jap mangan tsumo}. You can secure your position even if you deal into a {\jap mangan} hand by the South or the West players. However, if you {\jap riichi}, the North player can now get the first place with a {\jap mangan tsumo}, and the South player can get the first place with a direct hit {\jap mangan ron} from you. Do not run such risks by calling {\jap riichi}. Even when you get a ready hand with a really good wait, you must go {\jap dama}. For more discussions of what to do in South-4, see Chapter \ref{ch:grand}.

\subsection{{\jap Genbutsu} wait} \index{genbutsu@{\jap genbutsu}} \label{sec:genbutsu}
The third case where going {\jap dama} is OK is when another player is already getting a lot of attention from others (e.g., {\jap riichi}, {\jap dora} {\jap pon}, or {\jap honitsu}) and one of your winning tiles is among his {\jap genbutsu} tiles.
One player's {\jap genbutsu} tiles are all the tiles discarded by that player \emph{and} the tiles that are passed up by that player.\footnote{See Section \ref{sec:genbutsu_def} for a more detailed explanation.}

\bigskip
For example, suppose the dealer has the following hand and discards in East-1. 
\bp
\vspace{-20pt}
{\small Discards}\\ \vspace{-10pt}
\suo{4}\suo{2}\tong{2}\xi\tong{4}\suo{8}\\
\vspace{-10pt}
\hspace{-45pt}\tong{4}\tong{6}\suo{6}\wan{8}\\
\vspace{-10pt}
\ep
\bp
\suo{30}\suo{30}\suo{30}\suo{30}
\rbai\bai\bai~\wan{9}\rwan{9}\wan{9}~\dong\dong\rdong~~\suo{1}\\
\hspace{330pt}\footnotesize{\jap Dora}
\ep
This is a confirmed 7700 hand (seat \& prevailing Wind + White Dragon), and the hand value can easily go up to {\jap haneman} (18000) or {\jap baiman} (24000).\footnote{The hand can have (a combination of) the following {\jap yaku} in addition to what's already visible: {\jap toitoi} (All Pungs), Green Dragon, Red Dragon, {\jap dora}, {\jap honitsu}, {\jap honroutou} (All Terminals and Honors), and {\jap shousangen} (Little Three Dragons). With this hand, the maximum possible hand value is {\jap sanbaiman} (36000).} 
In such a situation, everybody will be paying attention to the dealer (as they should). Suppose further that you manage to make your hand ready for {\jap pinfu}, waiting for {\large\suo{3}-\suo{6}}. Then, you should keep the hand {\jap dama}, as {\large\suo{6}} is one of the dealer's {\jap genbutsu} tiles. There is a good chance that the other two players fold against the dealer completely and try to discard nothing but his {\jap genbutsu} tiles. 

\bigskip

Keep in mind, though, that there is a bad kind of attention as well. For example, suppose someone is doing the following.
\bp
\suo{30} \rwan{4}\wan{3}\wan{5}~\zhong\zhong\rzhong~\suo{9}\rsuo{9}\suo{9}~\rtong{4}\tong{2}\tong{3}
\ep
Suppose further he has already discarded {\jap dora}. Then, he is getting a lot of attention, but no one really cares about him, let alone folds against him. In such a situation, you should call {\jap riichi} even when your winning tiles are among his {\jap genbutsu}. Punish a player who makes bad calls like this.

\subsection{High scoring hand} \label{sec:high}
When your hand is already expensive without {\jap riichi} (minimum of 7700 if playing with red fives; 5200 if playing without a red five), it is OK to go {\jap dama}. Let's see a few examples. 

\bigskip
\begin{itembox}[r]{Expensive hand 1}
\bp
\wan{3}\wan{4}\wan{5}\tong{3}\tong{3}\tong{3}\tong{4}\tong{6}\tong{8}\suo{2}\suo{3}\suo{4}\suo{6}\suo{6}~~\tong{3}\\
\hspace{310pt}\footnotesize{\jap Dora}
\ep
\vspace{-15pt}
\end{itembox}
\noindent You should keep this hand {\jap dama} because the hand is already expensive ({\jap tanyao} + three {\jap dora} = {\jap mangan}) and the wait is not very good. 
\begin{center}
{\large $\Rightarrow$ Keep {\jap dama}, discarding the \tong{8}~!}
\end{center}
Why should we discard the {\large\tong{8}}, not the {\large\tong{4}} (which will make for a {\jap suji}-trap wait)? \index{suji@{\jap suji}!{\jap suji} trap}
This is to leave the possibility of improving the wait. If you draw or call {\jap pon} on a {\large\suo{6}} or the fourth {\large\tong{3}} after discarding the {\large\tong{8}}, the wait gets significantly better, as follows:
\bp
\wan{3}\wan{4}\wan{5}\tong{3}\tong{3}\tong{3}\tong{4}\tong{6}\suo{2}\suo{3}\suo{4}\suo{6}\suo{6}~~\tong{3}\\
\hspace{295pt}\footnotesize{\jap Dora}
\ep
\bp
\vspace{-50pt}
$\downarrow$\\
\wan{3}\wan{4}\wan{5}\tong{3}\tong{3}\tong{3}\tong{4}\suo{2}\suo{3}\suo{4} \rsuo{6}\suo{6}\suo{6}\\
\wan{3}\wan{4}\wan{5}\tong{3}\tong{4}\suo{2}\suo{3}\suo{4}\suo{6}\suo{6}
\tong{3}\rtong{3}\tong{3}\\
\ep

\bigskip
Note that, to justify {\jap dama} your hand has to have \emph{at least} 7700 when won by {\jap ron}. This means that (1) your hand has to have at least one {\jap yaku} (without it you cannot win by {\jap ron}) and (2) its value is at least 7700 when winning on a tile that gives you the lowest possible score. 

\bigskip
\begin{itembox}[r]{Expensive hand 2}
\bp
\wan{3}\wan{4}\wan{5}\tong{3}\tong{3}\tong{3}\tong{4}\tong{6}\tong{8}\suo{1}\suo{2}\suo{3}\suo{6}\suo{6}~~\tong{3}\\
\hspace{310pt}\footnotesize{\jap Dora}
\ep
\vspace{-15pt}
\end{itembox}
\noindent 
This hand does not have {\jap yaku} and so you cannot win by {\jap ron} without {\jap riichi}. You should do insta-{\jap riichi} with this hand by discarding the {\large\tong{4}}. 

\bigskip
\begin{itembox}[r]{Expensive hand 3}
\bp
\wan{3}\wan{4}\wan{4}\wan{6}\wan{6}\tong{3}\tong{4}\tong{5}\suo{3}\suo{4}\suo{5}\suo{6}\suo{7}\suo{8}~~\bei\\
\hspace{310pt}\footnotesize{\jap Dora}
\ep
\vspace{-15pt}
\end{itembox}

\noindent This hand is worth 7700 when winning on a {\large \wan{5}} ({\jap tanyao + pinfu + sanshoku}), but the hand is worth only 2000 if winning on a {\large \wan{2}} ({\jap tanyao + pinfu}). You should therefore do insta-{\jap riichi} by discarding a {\large\wan{4}}. 

\bigskip
When you have an expensive hand, going {\jap dama} is acceptable but calling {\jap riichi} is also an option, especially when you have a good wait. Consider the following hand. Should we {\jap riichi}?

\bigskip
\begin{itembox}[r]{Expensive hand 4}
\bp
\hspace{30pt}{\footnotesize\color{red!75!black} Red}\\ \vspace{-18pt}
\wan{3}\wan{4}\wan{6}\wan{6}\tong{2}\tong{3}\tong{4}\suo{4}\rfs\suo{6}\suo{6}\suo{7}\suo{8}\suo{9}~~\tong{3}\\
\hspace{310pt}\footnotesize{\jap Dora}
\ep
\vspace{-15pt}
\end{itembox}

\noindent Discarding the {\large\suo{9}}, we get a confirmed 7700 hand without {\jap riichi}. I already said above that it is acceptable to keep {\jap dama} when the minimum hand value is 7700. However, would calling {\jap riichi} be even better? 

\bigskip
{\jap Riichi} criteria for hands like this are as follows.

\bigskip
\begin{itembox}[c]{{\jap Riichi} judgement for an expensive hand}
\bi
\i {\jap Riichi} if you are far behind from other players.
\i {\jap Riichi} if it is the 6th turn or before and the wait is 2-way side wait or better.
\i {\jap Riichi} if it is the 10th turn or before and the wait is 3-way side wait or better.
\i Don't {\jap riichi} if the minimum hand value is {\jap haneman} or better.
\ei
\vsps
\end{itembox}

\newpage
\subsection{Many possibilities of improving the hand} \label{sec:imp}
It is OK to keep {\jap dama} when there are \emph{many} possibilities of further advancing your hand. Keep in mind, however, that it is rather rare that waiting in {\jap dama} is worthwhile; doing insta-{\jap riichi} is still better in most instances even when there are \emph{some} possibilities of advancing your hand. It makes sense to wait in {\jap dama} only when \emph{both} of the following two conditions are met. 

\bigskip
\begin{itembox}[c]{A reason to keep {\jap dama}: Improving the hand}
\bi
\i It is still an early turn (8th turn or before);
\i There are at least six kinds of tiles that can improve the scores and/or the wait, \\~~~~~or\\
there is at least one kind of tiles that can improve the score by at least three {\jap fan} in one step.
\ei
\end{itembox}

\bigskip
Keep in mind that waiting in {\jap dama} becomes less and less desirable towards the end of a hand. After passing the 9th turn (the midpoint of the middle row of your discards), you'd better call {\jap riichi} even if the second condition is met. Remember that the probability of drawing a particular tile is very small (roughly 3\%).\footnote{As I mentioned before, when there are four kinds of tiles that can improve your hand, it will take (on average) eight turns to draw one of them.}

\bigskip
If you decide not to {\jap riichi}, it often makes more sense to revert the hand to 1-away rather than maintaining a ready hand. As we learned in Chapter \ref{ch:basic}, a ready hand can accept fewer tiles than a 1-away hand can. For example, consider the following hand.
\bp
\wan{1}\wan{2}\wan{3}\wan{5}\wan{7}\wan{8}\wan{9}%
\tong{7}\tong{8}\tong{9}\tong{9}\tong{9}\suo{5}\suo{7}
\ep
Since calling {\jap riichi} by discarding the {\large \wan{5}} gives you {\jap riichi}-only with a bad wait, you may want to refrain from {\jap riichi}. However, if you discard the {\large \wan{5}}, the hand can be improved only if you draw a {\large \suo{4}} or a {\large \suo{8}}. Moreover, even when you luckily draw a {\large \suo{4}}, the hand is still only {\jap riichi} + {\jap pinfu}, albeit with an improved wait. 

\bigskip
A more sensible choice here is to discard the {\large \suo{5}} and revert the hand to 1-away, as follows. 
\bp
\wan{1}\wan{2}\wan{3}\wan{5}\wan{7}\wan{8}\wan{9}%
\tong{7}\tong{8}\tong{9}\tong{9}\tong{9}\suo{7}
\ep
This is another example of golden 1-away. \index{1-away (1-{\jap shanten})!golden 1-away}
This 1-away hand is so much better than the ready hand you'd get by discarding the {\large \wan{5}}. Specifically, there are four kinds of tiles that can improve the score by at least three {\jap fan}. 
\bi \index{sanshoku@{\jap sanshoku}}
\i If you draw a {\large\suo{9}}, the hand is ready for {\jap junchan} (Terminals in All Sets) + {\jap sanshoku} --- {\jap mangan} without {\jap riichi} and {\jap haneman} with {\jap riichi}. You may want to go {\jap riichi} in this case because your previous discard of {\large \suo{5}} makes for a {\jap suji} trap wait, although going {\jap dama} is also OK. \index{suji@{\jap suji}!{\jap suji} trap}
\i If you draw a {\large\suo{8}}, the hand is ready for {\jap pinfu + junchan + sanshoku} --- {\jap haneman} without {\jap riichi}. You may still want to {\jap riichi}. It would be a shame to win this monster hand on a {\large\suo{6}}, but doing so without {\jap riichi} is the worst. 
\i If you draw a {\large\wan{4}} or a {\large\wan{6}}, the hand is ready for {\jap pinfu + ittsu} --- 7700 with {\jap riichi}. You should definitely {\jap riichi}. Going {\jap dama} with this hand is absurd.
\ei

\bigskip

For another (less exciting) example, consider the following hand.
\bp
\wan{3}\wan{4}\wan{5}\wan{6}\suo{1}\suo{3}\suo{7}\suo{8}\suo{9}\tong{1}\tong{1}\tong{5}\tong{6}\tong{7}~~\bei\\
\hspace{315pt}\footnotesize{\jap Dora}
\ep
Suppose this is the 5th turn in a hand. The choice is between (1) discarding the {\large\wan{3}} to make this hand ready or (2) discarding the {\large\suo{1}} to revert the hand to 1-away. 

\bigskip
It is OK to choose either of the two in this case, but there is one thing you should \emph{not} do. That is discarding the {\large\wan{3}} without calling {\jap riichi}. 
If, according to your reading of the board, a closed wait of {\large\suo{2}} is good enough (e.g., none of {\large \suo{2}} has been discarded yet, but lots of other tiles in Bamboos have been discarded), do insta-{\jap riichi}. Waiting in {\jap dama} with a {\jap yaku}-less hand is generally a bad move. If you discard the {\large\wan{3}} just to keep the hand ready, the hand cannot be won by {\jap ron}, and it can be improved only if you draw a {\large\suo{4}}. It makes no sense to have such a hand.\footnote{Of course, if this were towards the very end of a hand (15th--18th turn), it would make a lot of sense to have a {\jap yaku}-less {\jap dama} hand in order to avoid {\jap noten} penalty.} 

\bigskip
If you want to wait and improve the hand, you should discard the {\large\suo{1}} to revert the hand to 1-away. If you draw any of {\large\wan{2}\wan{4}\wan{5}\wan{7}\suo{4}} (5 kinds--15 tiles), both the wait and the scores get improved. If you draw any of {\large\wan{3}\wan{6}\suo{3}\tong{1}} (4 kinds--11 tiles), at least the wait gets improved. 
A basic rule of thumb with a {\jap yaku}-less hand is as follows.

\bigskip

\begin{itembox}[c]{What to do with a {\jap yaku}-less hand}
\bi
\i {\jap Riichi} if you make the hand ready.
\i Don't make it ready if you don't {\jap riichi}.
\vsps
\ei
\end{itembox}

\subsubsection*{{\jap Yaku}-less {\jap dama}}
In some exceptional situations, it may make sense to wait in {\jap dama} while keeping a {\jap yaku}-less ready hand. 
\bp
\wan{3}\wan{4}\wan{5}\wan{8}\tong{3}\tong{4}\tong{5}\tong{9}\suo{1}\suo{2}\suo{3}\suo{6}\suo{7}\suo{8}~~\suo{3}\\
\hspace{315pt}\footnotesize{\jap Dora}
\ep
This hand has one {\jap dora} with a bad wait (single wait of {\large \tong{9}} or {\large \wan{8}}). According to the {\jap riichi} criteria given at the beginning of this chapter, you could call {\jap riichi} with this hand. In fact, if it is already the 9th turn or later, you should definitely do insta-{\jap riichi}. However, if it is the 6th turn or before, waiting in {\jap dama} by discarding the {\large \tong{9}} or the {\large\wan{8}} is an option. 
If you draw any of {\large\wan{3}\wan{5}\tong{3}\tong{5}\suo{1}\suo{6}\suo{8}} (7 kinds--21 tiles), the hand becomes {\jap pinfu} with a semi side wait. Moreover, with {\large\wan{2}\wan{6}\tong{2}\tong{6}\suo{4}\suo{5}\suo{9}} (7 kinds--28 tiles) the wait will become a stretched single wait. When there are this many possibilities of improving a hand (14 kinds--49 tiles), it is OK to go {\jap dama} with a {\jap yaku}-less hand.

\newpage
\section{Glossary}

\begin{description}
\item[Insta-{\jap riichi}] is to {\jap riichi} immediately when a hand becomes ready rather than wait for a few turns to {\jap riichi}. Basically, all {\jap riichi} should be insta-{\jap riichi}.
	\index{insta-{\jap riichi}}
\item[{\jap Dama}] is not to {\jap riichi} when having a closed ready hand. See Section \ref{sec:dama} for cases where going {\jap dama} might be better than {\jap riichi}. 
	\index{dama@{\jap dama}}
\item[{\jap Yaku}-less {\jap dama}] is when you have a ready hand with no {\jap yaku} and choose not to {\jap riichi}. There are very few instances where doing so is justifiable. 
\item[{\jap Genbutsu}] are tiles that are safe for a given player, either because they were discarded by that player or because they are discarded by other players after that player called {\jap riichi}. See Section \ref{sec:genbutsu_def} in Chapter \ref{ch:defense}.
	\index{genbutsu@{\jap genbutsu}}
\end{description}

