%~~~~~~~~~~~~~~~~~~~~~~~~~~~~~~~~~~~~~~~~~~~~~~~~~
% Riichi Book 1, Chapter 10: Grand
%~~~~~~~~~~~~~~~~~~~~~~~~~~~~~~~~~~~~~~~~~~~~~~~~~

\chapter{Grand strategies} \label{ch:grand}
\thispagestyle{empty}
The most important goal in mahjong is to win a game or generally improve the placement. I do not deny the inherent joy of winning an expensive hand with rare {\jap yaku}. However, we should always keep in mind that winning a hand is just a means to an end; sometimes dealing into an opponent's (cheap) hand can serve our purpose of winning the game. 
In this chapter, I will discuss strategies to improve the placement.

\section{What do do in South-4}
Most mahjong rule sets adopt some type of {\jap uma} system where players get some extra bonus / penalty points according to the placement. For example, EMA rules award $15000$ points to the first ranked player, $5000$ points to the second ranked player, $-5000$ points to the third ranked player, and $-15000$ points to the fourth ranked player. \index{european@EMA}
Such systems make it clearer that getting a better placement is generally more important than simply winning hands. 

\bigskip
Suppose you are currently ranked fourth in South-4, and that the third ranked player has 1800 more points than you do. In such a situation, winning a 1000 hand is not very meaningful. If you manage to add just one more {\jap han} and win, you will not only get 2000 points directly but also get an extra 10000-point bonus for coming in third,\footnote{You will get $-5000$ points instead of $-15000$ points, resulting in a net gain of $10000$.} a total of 12000-point gain. This is as big as winning a {\jap haneman} hand. 

\bigskip
In a situation like this, the tradeoff between speed and hand value is qualitatively different than usual. For example, when choosing between a good-wait one-{\jap han} hand and a bad-wait two-{\jap han} hand, you should definitely choose the latter. After all, you are essentially comparing a good-wait 1000-point hand with a bad-wait 12000-point hand. On the other hand, when choosing between a good-wait two-{\jap han} hand and a bad-wait three-{\jap han} hand, you should choose the former. Increasing the (virtual) hand value from 12000 to 13900 would not be worthwhile if doing so significantly diminishes the chance of winning. 

\bigskip
Suppose further that the third ranked player is the dealer. Then, you will have another option to improve the placement. That is, if anyone other than the third ranked player gets a {\jap mangan tsumo} (or above), you will come in third. This is because the dealer pays 2000 more points than a non-dealer in case of a {\jap mangan tsumo}. When this happens, you will lose 2000 points for the {\jap mangan} payment but gain 10000 points for the placement bonus, resulting in a net gain of 8000 points. This is as big as winning a {\jap mangan} hand yourself.

\bigskip
Suppose yet further that the second ranked player is the right player, who is behind the first ranked player by 10000 points. Then, he will try to get a {\jap mangan tsumo} because doing so puts him in the first place. If he is obviously pursuing a {\jap honitsu} hand, you may want to discard tiles in the suit he is collecting so he can meld his hand to get it ready.\footnote{Of course, you assist him only until he gets ready. You need to be careful not to deal into the {\jap mangan} hand you helped him make.} If he indeed gets a {\jap mangan tsumo}, you will get 8000 points; if he wins by {\jap ron} from the third ranked player, you will get 10000 points. 

\bigskip
That being said, assisting other players in hopes of their getting a {\jap mangan tsumo} is more like a last resort. What you should think about first and foremost is winning your own hand that is just expensive enough to improve your placement, which we will now turn to.

\subsection*{Improving the placement by {\jap ron} / {\jap tsumo}}
As the discussion in the previous section illustrates, you need to be extra conscious about your placement in South-4. If you are currently ahead of the game, your top priority is to maintain your placement. If you are behind, you should do your best to improve your placement as much as possible. 

\bigskip
In South-4, the first thing you need to do \emph{before} the hand begins is to figure out the point differences between you and other players. When playing online on {\jap Tenhou}, this can be easily done any time by mouseovering the middle board, as illustrated in Section \ref{sec:indicator}. When playing offline, each player should count and report their points before the hand begins. 

\bigskip
Once you figure out the point differences, you then need to know how expensive your hand has to be to improve your placement. In doing so, you need to figure out the required hand values under three possibilities, as follows.
\be
\i {\jap ron} from anyone
\i {\jap tsumo}
\i direct hit {\jap ron}
\ee
The first possibility to consider is winning your hand by {\jap ron} from anyone (that is, not from the very player you are trying to overtake). For example, suppose you are currently ranked second, and the first ranked player has 3400 more points. Then, winning a 3 {\jap han}--30 minipoints hand (3900 points) by {\jap ron} from anyone is sufficient to improve your placement. You should thus aim to have a 3-{\jap han} hand. Since you don't need a 40-minipoint hand, melding is also an option. 

\bigskip
The second possibility to consider is winning your hand by {\jap tsumo}. For example, suppose you are currently ranked second, and the first ranked player (non-dealer) has 9500 more points. Then, getting a {\jap mangan tsumo} is sufficient to improve your placement because you gain 8000 points while the first ranked player loses 2000 points, inducing a 8000 + 2000 = 10000 point difference. You should thus aim to have a {\jap mangan} hand and try to win it either by {\jap tsumo} or by {\jap ron} from the first ranked player. 

\bigskip
The last possibility to consider is winning your hand by {\jap ron} from the very player you are trying to overtake. For example, suppose you are currently ranked second, and the first ranked player (non-dealer) has 15200 more points. Then, even getting a {\jap haneman tsumo} is not enough. You need a {\jap mangan ron} directly from the first ranked player. This is sufficient because you gain 8000 points while the first ranked player loses 8000 points, inducing a 8000 $\times$ 2 = 16000 point difference. 

\bigskip
Among these three possibilities, your hand value judgement should be based primarily on the first possibility (i.e., winning it by {\jap ron} from anyone). This one requires the highest hand value but it is the most realistic. Given that the player you are trying to overtake will try hard not to deal into your hand, making your hand value judgement based solely on the third possibility (direct hit {\jap ron}) is too much of a wishful thinking. 

\bigskip
With this in mind, consider the following hand. Assume that you are the North player in the 6th turn in South-4. You are currently ranked second, and the first ranked player (South) has 5100 more points. 

\bigskip
\begin{itembox}[r]{South-4: 5100 points behind, {\jap dora} {\Large\wan{6}}}
\bp
\wan{1}\wan{2}\wan{3}\wan{6}\tong{1}\tong{2}\tong{3}\tong{4}\tong{4}\suo{2}\suo{4}\suo{5}\suo{6}~~\wan{8}\\
\hfill\footnotesize{Draw~~~~~~~~~~}
\ep
\vspace{-17pt}
What would you discard?
\end{itembox}
\noindent If you keep {\LARGE\wan{8}} and discard {\LARGE\suo{2}}, the hand is ready. However, doing insta-riichi with the current hand is not ideal. Since the hand value is only 2600 (2 {\jap han}--40 minipoints), winning it by {\jap ron} from the third-ranked or fourth-ranked player will not improve your placement (unless you get {\jap ippatsu} or {\jap ura dora}). Also, getting {\jap tsumo} will only give you 1000-2000 (3 {\jap han}--30 minipoints), generating only a 4000 + 1000 = 5000 point difference. This is not sufficient to improve the placement. 

\bigskip
You should rather keep the hand 1-away by discarding {\LARGE\wan{8}}. If you draw {\LARGE\suo{1}} or {\LARGE\suo{3}}, you can then do insta-riichi to get riichi + {\jap sanshoku} = at least 5200 (3 {\jap han}--40 minipoints). Winning it by {\jap ron} from anyone is now sufficient to improve the placement. If you draw {\LARGE\wan{5}} or {\LARGE\wan{7}}, you can also do insta-riichi to get riichi + {\jap pinfu} + {\jap dora}. Winning it either by {\jap tsumo} or {\jap ippatsu ron} is sufficient to improve the placement.\footnote{Drawing \wan{7} means you are in {\jap furiten}, but you should still do insta-riichi.}

\subsection*{Point difference induced by {\jap tsumo}}
Getting the correct point differences induced by {\jap tsumo} can be a bit complicated. For example, suppose you are the North player, currently ranked second in South-4. The West player is leading the game, having 6300 more points. In this situation, would winning a 3 {\jap han}--30 minipoints hand by {\jap tsumo} be enough to improve the placement? What about winning a 3 {\jap han}--40 minipoints (= 4 {\jap han}--20 minipoints) hand by {\jap tsumo}? 

\bigskip
To calculate the point difference induced by {\jap tsumo}, we add the points you gain and the points your rival (the first ranked player) loses. For instance, the point difference induced by a 3 {\jap han}--30 minipoints hand is: 4000 (your gain) + 1000 (your rival's loss) = 5000 points. The point difference induced by a 3 {\jap han}--40 minipoints hand is: 5200 (your gain) + 1300 (your rival's loss) = 6500 points. 
In this example, getting a 3 {\jap han}--40 minipoints {\jap tsumo} is sufficient to improve the placement, but getting a 3 {\jap han}--30 minipoints {\jap tsumo} is not. 

\bigskip
It would be extremely tedious if we have to do these calculations for several possible hand values all in our head in South-4. It would be more efficient if we memorize the induced point differences for several representative cases; that way, we can use our time and energy thinking about other important things during the game.

\bigskip
Tables \ref{tbl:pd1}--\ref{tbl:pd4} below summarize induced point differences for limit hands and those with 30, 40 (20), and 50 (25) minipoints. 
In each table, the second column shows the induced point differences against another non-dealer, whereas the third column shows those against the dealer. Since a dealer pays twice as much as a non-dealer, the induced point differences against a dealer are greater. 
In addition, for each counter (continuation) placed on the table, the induced point difference will get bigger by 400 points. 

{\begin{table}[t!]
\centering\captionsetup{font=small}\small
\begin{minipage}[h]{0.48\hsize}
\caption{Limit hands} \label{tbl:pd1}
\begin{tabular}{r r r}
\toprule
{\jap Tsumo} & {\footnotesize Non-dealer} & {\footnotesize  Dealer}\\
\midrule
{\jap mangan}	&	10000	&12000\\
{\jap haneman}	&	15000	&18000\\
{\jap baiman}	&	20000	&24000\\
{\jap yakuman}	&	40000	&48000\\
\bottomrule
\end{tabular}
\end{minipage}
    \hfill
\begin{minipage}[t!]{0.48\hsize}\centering
\caption{30 minipoints}\label{tbl:pd2}
\begin{tabular}{r r r}
\toprule
{\jap Tsumo} & {\footnotesize Non-dealer} & {\footnotesize  Dealer}\\
\midrule
300-500	&	1400	& 1600\\
500-1000 &	2500	& 3000\\
1000-2000 &	5000	& 6000\\
2000-3900 &	9900 & 11900\\
\bottomrule
\end{tabular}
\end{minipage}
\end{table}}

{\begin{table}[t!]
\centering\captionsetup{font=small}\small
\begin{minipage}[h]{0.48\hsize}
\caption{40 (20) minipoints}\label{tbl:pd3}
\begin{tabular}{r r r}
\toprule
{\jap Tsumo} & {\footnotesize Non-dealer} & {\footnotesize  Dealer}\\
\midrule
400-700	&	1900	&2200\\
700-1300	&	3400&4000\\
1300-2600&	6500&7800\\
\bottomrule
\end{tabular}
\end{minipage}
    \hfill
\begin{minipage}[h]{0.48\hsize}\centering
\caption{50 (25) minipoints}\label{tbl:pd4}
\begin{tabular}{r r r}
\toprule
{\jap tsumo} & {\footnotesize Non-dealer} & {\footnotesize  Dealer}\\
\midrule
800-1600	&	4000	& 4800\\
1600-3200 &	8000	& 9600\\
\bottomrule
\end{tabular}
\end{minipage}
\end{table}}

\bigskip
Note that these four tables assume that you are a non-dealer.
When you are the dealer, you do not usually need to do these calculations because you get to continue the game if you win a hand anyway. However, when playing with a bankruptcy rule or with time limits, the dealer may not be able to continue the game, in which case even the dealer has to consider if winning a particular hand improves the placement. Tables \ref{tbl:pd5} and \ref{tbl:pd6} at the end of this chapter provide a summary for a dealer as well.

\bigskip
Memorizing these tables would be \emph{way} more important than memorizing, say, scores for 70-minipoint hands. With these tables in mind, consider the following hand. Assume that you are the North player in the 6th turn in South-4. You are currently ranked second, and the first ranked player (South) has 3300 more points. 

\bigskip
\begin{itembox}[r]{South-4: 3300 points behind, {\jap dora} {\LARGE\wan{5}}}
\bp
\wan{1}\wan{1}\wan{5}\wan{7}\wan{8}\tong{3}\tong{4}\tong{5}\tong{6}\tong{7}\tong{8}\suo{2}\suo{3}~~\suo{1}\\
\hfill\footnotesize{Draw~~~~~~~~~~}
\ep
\vspace{-17pt}
What would you discard?
\end{itembox}
\noindent 
You wanted to draw {\LARGE\wan{6}} first so that you can have riichi + {\jap pinfu} + one {\jap dora} = 3900. Winning that hand by {\jap ron} from anyone would improve your placement. However, now that you drew {\LARGE\suo{1}}, what should you do?

\bigskip
Recall that a 700-1300 {\jap tsumo} would induce a 3400 point difference. Since this is a {\jap pinfu} hand, getting {\jap riichi + pinfu + tsumo} gives you exactly 700-1300 {\jap tsumo}. You should thus do insta-riichi by discarding {\LARGE\wan{5}}. Once you call riichi, you can do either (1) {\jap ippatsu ron} from anyone, (2) direct hit {\jap ron} from the first ranked player, or (3) {\jap tsumo} to improve the placement.\footnote{Whether or not you should let it go when the third or fourth ranked player discards your winning tile depends on the point difference between you and the third ranked player. Unless it is greater than 12000 points, you should call {\jap ron} and hope to get one {\jap ura dora}.}

\bigskip
Consider a more complicated example that involves some minipoint calculation. Assume that you are the North player in the 6th turn in South-4. You are currently ranked second, and the first ranked player (East) has 4700 more points. 
What are the conditions under which you can improve your placement with the following hand?

\bigskip
\begin{itembox}[r]{South-4: 4700 points behind, {\jap dora} {\LARGE\wan{5}}}
\bp
\wan{2}\wan{2}\wan{2}\suo{4}\suo{4}\fa\fa~\rtong{8}\tong{8}\tong{8}~\suo{7}\suo{7}\rsuo{7}
\ep
\vspace{-10pt}
What are you waiting for?
\end{itembox}
\noindent Winning this hand on {\LARGE\fa} by {\jap ron} from anyone or {\jap tsumo} satisfies the condition because it gives you 5200 points ({\jap toitoi} + Green Dragon with 40 minipoints). Winning it on {\LARGE\suo{4}}, however, only gives you 2600 points. You can still improve the placement if you get a direct hit from the first ranked player, but not if it is from other players. Even though the first ranked player is the dealer, you cannot improve the placement if you draw {\LARGE\suo{4}}, either. Declaring {\jap tsumo} on {\LARGE\suo{4}} gives you 700-1300, which induces only a 4000 point difference even against the dealer. 

\bigskip
However, if you manage to draw {\LARGE\tong{8}} or {\LARGE\suo{7}}, you should extend the melded set to a melded quad. Doing so not only gives you a chance of {\jap rinshan tsumo} or new {\jap dora} but also enables you to improve the placement when drawing {\LARGE\suo{4}}. This is because the hand will have 50 minipoints if you {\jap tsumo}: 20 for the base minipoints + 8 for a melded Kong of {\LARGE\tong{8}} or {\LARGE\suo{7}} + 2 for a melded set of {\LARGE\suo{7}} or {\LARGE\tong{8}} + 4 for a concealed set of {\LARGE\wan{2}} + 2 for a pair of {\LARGE\fa} + 4 for a concealed set of {\LARGE\suo{4}} + 2 for {\jap tsumo} = 42, rounded up to 50 minipoints. A 2 {\jap han}--50 minipoints {\jap tsumo} induces a 4800 point difference against the dealer.

%\newpage
\subsection*{Maintaining your placement}
If you are ahead of the game in South-4, you should do your best to maintain your current rank. Trying to win a cheap but fast hand to end the game is an option, but be extra careful not to deal into an opponent's expensive hand. For example, suppose you have 15200 more points than the second ranked player. If neither you nor the second ranked player is the dealer, he cannot defeat you even with a {\jap haneman tsumo}. Then, what you need to be wary of the most is to give him a direct hit {\jap mangan ron}. You will lose not only the 8000 points for the {\jap mangan} payment but also the 10000 bonus points for the placement, a total net loss of 18000 points.

\bigskip
In order to figure out what exactly you should do when you are ahead of the game in South-4, try to imagine what each of your opponents aims to do. Recall the situation I described in discussing riichi judgement in Section \ref{sec:dama}, reproduced below.

\begin{table}[h]
\begin{center}
\begin{tabular}{l r l r}
East (you) & 39000 & South & 22900\\
West & 13000 & North & 25100\\
\end{tabular}
\end{center}
\vspace{-10pt}
\end{table}

\bigskip
Let's think about the incentive structure for each of the other three players in turn. First, the fourth ranked player (West) should try to have a {\jap mangan tsumo}, for that would put him in the third place. The third ranked player (South) would need a 500-1000 {\jap tsumo} or 2600 {\jap ron} to get the second place, which is a realistic goal to pursue. In order for him to get the first place, he would need either a {\jap haneman tsumo} or a direct hit {\jap haneman ron} from you. Finally, in order for the second ranked player (North) to get the first place, he would need a {\jap haneman tsumo} or a direct hit {\jap mangan ron} from you. Given that he has only 2200 more points than the third ranked player, the second ranked player should rather aim to win whatever hand possible to maintain the current rank. 

\bigskip
So, what should you do in such a situation? 
What you should be afraid of the most is a {\jap haneman tsumo} by South or North. However, notice that South and North are in a fierce competition among themselves. Take advantage of this. If winning a fast hand yourself does not seem possible, you should try to assist the South player. Since the South player is your right player, you should discard versatile middle tiles (3--7) so that he would call {\jap chii} on them, possibly with a red five (because South needs 2 {\jap han}). Recall that even giving him a direct hit {\jap mangan ron} will secure you the first place. 

\section{What to do by South-3}
It is never too early to start paying attention to your placement. If you are behind other players, the target point difference you should achieve before the beginning of South-4 is 10000 points or fewer. Overtaking with a 10000 point difference in a single hand is a tough but not unrealistic goal; you can do so either by {\jap mangan tsumo} or {\jap haneman ron}. 

\bigskip
Suppose the dealer wins a {\jap mangan tsumo} in East-1. Now he is leading other players by 16000 points, which is a bit depressing. However, instead of trying to overtake him with a single expensive hand, you should aim to reduce the point difference from 16000 to 10000 by the end of South-3. This is a more realistic goal; keep calling riichi with a {\jap pinfu}-only hand, and you will eventually get {\jap tsumo} + one {\jap ura dora} which induces a 6500 point difference. 

\bigskip
On the other hand, if you are ahead of the game, you should aim to have a 10000 or greater point difference with the second ranked player in South-4. 
For example, consider the following hand. Suppose you are the North player in the 6th turn in South-3. You are currently ranked first, and you only have 1000 more points than the second ranked player (West). 

\bigskip
\begin{itembox}[r]{{\jap Riichi} judgement in South-3: {\jap dora} {\LARGE\wan{6}}}
\bp
\wan{2}\wan{3}\wan{4}\wan{6}\wan{7}\wan{8}\tong{3}\tong{3}\tong{3}\suo{4}\suo{5}\suo{7}\suo{7}\bei
\ep
\vspace{-10pt}
{\jap Riichi} or {\jap dama}?
\end{itembox}
\noindent The choice here is between keeping the hand {\jap dama} to maximize your chance of winning the hand or calling riichi to aim for a bigger point difference. 
You should do insta-riichi in such a situation. If you win this hand by {\jap dama ron}, the point difference will only be 3600 in South-4. Having a 3600 point difference is not much different from having a 1000 point difference from the perspective of the second ranked player. However, if you win this hand with riichi, the point difference will be at least 6200 in South-4, 9000 if you get one {\jap ura dora}, and 11000 if you {\jap tsumo}. Having a point difference of 11000 in South-4 significantly increases your chance of winning the game. 

\bigskip
If you will be the dealer in South-4, the target point difference with the second ranked player is 12000 or more, not 10000. This is because a {\jap mangan tsumo} by a non-dealer induces a 12000 point difference against the dealer. 

\bigskip
You should also be mindful of induced point differences by {\jap noten} penalties in South-4 and South-3. The maximum point difference induced by {\jap noten} penalties is 4000 (1-player {\jap noten} and 3-player {\jap noten}). This means that you should aim to have at least 4000 point difference by the end of South-3. 
For example, suppose you are leading the game in South-4 and you are the dealer. If the hand ends in exhaustive draw and the point difference between you and the second ranked player is more than 4000 points. Then, you should declare {\jap noten} (not ready) and terminate the game \emph{even when} you have a ready hand. You will not have this option if the point difference is fewer than 4000 points. 

\newpage

\section{Tables for induced point differences}

{\begin{table}[h!]
\centering\captionsetup{font=small}\small
\begin{minipage}[h]{0.49\hsize}
\caption{For non-dealer} \label{tbl:pd5}
\begin{tabular}{l r r}
\toprule
{ Tsumo} & {\footnotesize Non-dealer} & {\footnotesize Dealer}\\
\midrule
300-500 &	1400	& 1600\\
400-700 &	1900	&2200\\
500-1000&2500&3000\\
700-1300&3400&4000\\
800-1600&4000&4800\\
1000-2000&5000&6000\\
1300-2600&6500&7800\\
1600-3200&8000&9600\\
2000-3900&9900&11800\\
2000-4000&10000&12000\\
3000-6000&15000&18000\\
4000-8000&20000&24000\\
6000-12000&30000&36000\\
8000-16000&40000&48000\\
\bottomrule
\end{tabular}
\end{minipage}
    \hfill
\begin{minipage}[h]{0.49\hsize}\centering
\caption{For dealer} \label{tbl:pd6}
\begin{tabular}{l r}
\toprule
{Tsumo} & {\footnotesize Non-dealer} \\
\midrule
500&	2000\\
700&	2800\\
1000&4000\\
1300&5200\\
1600&6400\\
2000&8000\\
2600&10400\\
3200&12800\\
3900&11600\\
4000&16000\\
6000&24000\\
8000&32000\\
12000&48000\\
16000&64000\\
\bottomrule
\end{tabular}
\end{minipage}
\end{table}}

