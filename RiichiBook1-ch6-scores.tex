%~~~~~~~~~~~~~~~~~~~~~~~~~~~~~~~~~~~~~~~~~~~~~~~~~
% riichi Book 1, Chapter 6: Scoring
%~~~~~~~~~~~~~~~~~~~~~~~~~~~~~~~~~~~~~~~~~~~~~~~~~

\chapter{Scoring}\label{ch:scores}
\thispagestyle{empty}

The scoring system in mahjong is quite complex. Getting proficiency in score calculation requires a lot of practice. The good news is that scoring is automatically done once you win a hand when you play online. Even when you play offline, you can usually count on your fellow players to help you get the correct score once you win a hand. 

\bigskip
However, you often need to calculate the (potential) scores of your hand \emph{before} you win the hand. This is because a lot of important judgements you make during the game --- riichi judgement, defense judgement, and melding judgement, among others --- depend on the potential scores of your hand. Therefore, developing ability to calculate the scores correctly and quickly without any help of others is of utmost importance. I introduce some efficient methods of score calculation in this chapter before we discuss riichi, defense, and melding judgements in the subsequent chapters. 

\section{Three steps in score calculation}

Every rule book of mahjong has comprehensive scoring tables (similar to Tables \ref{tbl:scores1} and \ref{tbl:scores2} at the end of this chapter) that show all possible scores for all possible minipoints ({\jap fu}). Although such tables are a good reference to have, it is \emph{not} very efficient to try to memorize everything in such tables. 

\bigskip
A more practical approach would be to focus on a small number of frequently observed patterns of scoring and memorize them correctly, while ignoring other, less important ones. Before introducing some shortcuts to do efficient scoring, let's first review the three required steps in score calculation, summarized in a box on the next page.
\bigskip

\begin{itembox}[c]{Three steps in score calculation}
\begin{description}
\i[Step 1:] {\bf Count the number of {\jap han}.}\\ 
First, you need to figure out how many {\jap han} a hand has. If a hand has five or more {\jap han}, skip Step 2 and go directly to Step 3. If not, proceed to Step 2. 

\i[Step 2:] {\bf Figure out the minipoints.}\\ 
When a hand has four or less {\jap han}, you then need to know the hand's minipoints. This does not mean, however, that you always need to do some maths to get the correct minipoints. We will discuss some practical shortcuts below.

\i[Step 3:] {\bf Get the score.}\\
Based on the number of {\jap han} (and possibly minipoints), you get the score. You will have to memorize some score patterns.
\end{description}
\end{itembox}

\bigskip
In the remainder of this chapter, I will first introduce basic methods of score calculation in Section \ref{sec:scores1}. The basic methods involve using some shortcuts in Step 2 above. Once you master the basic methods, you will be able to calculate scores correctly most of the time.\footnote{In my impression, roughly 75 \% of the hands we encounter can be covered by the basic methods.} When you master the contents of Section \ref{sec:scores1}, you may skip Section \ref{sec:scores2} and proceed to the next chapter. Section \ref{sec:scores2} covers more advanced methods of score calculation, which would be necessary only in exceptional cases. This involves an exact calculation of minipoints in Step 2 above. 


\section{Basic scoring}\label{sec:scores1}

\subsection*{1. Counting the number of {\jap han}}
Step 1 in score calculation is counting the number of {\jap han} in a hand. This is the most important part in score calculation, and there is no useful shortcut here. You need to be able to identify all the {\jap yaku} in a hand as well as the associated {\jap han} counts for each. 

\bigskip

A good way to practice this is to try to beat the automatic score counting on {\jap Tenhou}. Whenever someone wins a hand, {\jap Tenhou} displays all the {\jap yaku} and the associated {\jap han} counts one after another in a few seconds. Try to identify all the {\jap yaku} of your opponent's hand before they get displayed automatically. 

\subsubsection{Scores for limit hands} 
When a hand has five or more {\jap han}, the hand is a limit hand. Scores of limit hands do not depend on minipoints so we can go directly to Step 3. To get the score for a given {\jap han} count, we utilize Table \ref{tbl:limithands}. This is something you need to memorize. 

{\begin{table}[h!]
\centering\small \captionsetup{font=small}
\caption{Scores for limit hands} \label{tbl:limithands}
\begin{tabular}{llrrrr}
\toprule
{\jap Han} & Name & \multicolumn{2}{c}{{\jap Ron}}& \multicolumn{2}{c}{{\jap Tsumo}}\\
&&\multicolumn{1}{c}{\footnotesize Non-dealer}&\multicolumn{1}{c}{\footnotesize Dealer}&\multicolumn{1}{c}{\footnotesize Non-dealer}&\multicolumn{1}{c}{\footnotesize Dealer}\\
\midrule
5 & {\jap mangan} & {\bf 8000} & 12000  & 2000-4000 & 4000-all\\ [\sep]
6--7 & {\jap haneman} & 12000 & 18000  & 3000-6000 & 6000-all\\ [\sep]
8--10 & {\jap baiman} & 16000 & 24000  & 4000-8000 & 8000-all\\ [\sep]
11--12 & {\jap sanbaiman} & 24000 & 36000  & 6000-12000 & 12000-all\\ [\sep]
13+$^*$ & {\jap yakuman} & 32000 & 48000  & 8000-16000 & 16000-all\\
\bottomrule
\multicolumn{6}{l}{\footnotesize $^*$ A hand with 13+ {\jap han} is scored as a {\jap sanbaiman} with the revised EMA rules.}
\end{tabular}
\end{table}}
As you can see, there are some regularities and redundancies that make it relatively easy to memorize this table. The most important score of all is 8000 ({\jap mangan ron} for non-dealer). This is the basis of all the other scores in this table. For example, {\jap haneman} scores are 1.5 times {\jap mangan} scores, {\jap baiman} scores are two times {\jap mangan} scores, {\jap sanbaiman} scores are three times {\jap mangan} scores,\footnote{Since {\jap bai} means ``twice'' or ``double'' in Japanese, {\jap baiman} literally means double {\jap mangan} in Japanese. Similarly, {\jap sanbai} means ``triple,'' so {\jap sanbaiman} literally means triple {\jap mangan}.} and {\jap yakuman} scores are four times {\jap mangan} scores.
In addition, scores for dealer are exactly 1.5 times the corresponding scores for non-dealer in all limit hands. Finally, scores for {\jap tsumo} (self draw) cases are simple and straightforward; the dealer pays one half of the total, and each of the two non-dealers pays one fourth of the total. For example, in the case of {\jap mangan tsumo} (8000), the dealer pays 4000 and non-dealers pay 2000 each.

\subsection*{2. Figuring out the minipoints} 
	\index{fu@{\jap fu} (minipoint)} \index{minipoint ({\jap fu})}
When a hand has four or less {\jap han}, you have to know the minipoints. As I pointed out earlier, this does not mean that you always have to count all the minipoint contributions from all the melds and wait in a hand. Such a calculation is required only in special cases. Instead, you can use the chart in Figure \ref{fig:mini} that summarizes the six most frequently observed patterns you need to memorize. 

\bigskip

\begin{figure}[h!]
\begin{itembox}[c]{Shortcut for minipoint calculation}
\be
\i {\jap chiitoitsu} $\Rightarrow$ {\bf always} 25 minipoints
\i A hand has one or more quads $\Rightarrow$ Don't bother. %Ask someone.
\i {\jap toitoi} $\Rightarrow$ {\bf almost always} 40 minipoints 
%\\~~~~~~~~~~~~~~(unless {\jap tanyao})
\i {\jap Pinfu}
	\bi 
	\i {\jap ron} $\Rightarrow$ {\bf always} 30 minipoints
	\i {\jap tsumo} $\Rightarrow$ {\bf always} 20 minipoints
	\ei
\i Closed hand without {\jap pinfu}
	\bi 
	\i {\jap ron} $\Rightarrow$ {\bf almost always} 40 minipoints
	\i {\jap tsumo} $\Rightarrow$ {\bf almost always} 30 minipoints
	\ei
\i Open hand $\Rightarrow$ {\bf almost always} 30 minipoints
\ee
\end{itembox}
\caption{Six most observed patterns}\label{fig:mini}
\end{figure}

%\bigskip
You can use this Figure as a flowchart. 
You first check if the hand is a {\jap chiitoitsu} (Seven Pairs) hand. If it is, it is always 25 minipoints. If it is not, you then check if the hand has one or more quads (kongs). If it does, the hand is out of the scope of the basic methods. Ask for help from more experienced players after winning the hand. Advanced methods we discuss in Section \ref{sec:scores2} will cover this exceptional case. 
Third, you check if the hand is {\jap toitoi} (All Pungs).\footnote{Note that we are only talking about open {\jap toitoi} here. A closed {\jap toitoi} does not require a minipoint calculation under any circumstance. If you win it by {\jap tsumo}, it's {\jap yakuman} ({\jap su anko}; Four Concealed Pungs); if you win it by {\jap ron}, it's at least {\jap mangan} ({\jap toitoi} and {\jap san anko}).} If it is, it's almost always 40 minipoints unless it is a {\jap tanyao toitoi} hand that is likely to have 30 minipoints.  

\bigskip
Once you rule out the first three cases ({\jap chiitoitsu}, quads, and {\jap toitoi}), the last three are the most important ones; the great majority of hands you see will be one of these three. Here, you check two things. 
First, check if it is a {\jap pinfu} hand. If it is, it's always 30 minipoints ({\jap ron}) or 20 minipoints ({\jap tsumo}). 
If it is not {\jap pinfu}, you then check if it is a closed hand or an open hand. If it is a closed hand, it is \emph{almost} always 40 minipoints ({\jap ron}) or 30 minipoints ({\jap tsumo}). If it is an open hand, it is \emph{almost} always 30 minipoints whether you win it by {\jap ron} or {\jap tsumo}. 

\bigskip
Because of the importance of the cases 4, 5, and 6 ({\jap pinfu}, closed, and open hand), we will first discuss these three cases. We will then discuss cases 1 and 3 ({\jap chiitoitsu} and {\jap toitoi}), which are way more exceptional. 

\subsubsection{What do we mean by ``almost always''?}
Before getting to score calculation, let me explain what we mean when we say something is \emph{almost} always X minipoints in cases 5 and 6 in Figure \ref{fig:mini}. 
With a non-{\jap toitoi} hand without quads, check if the hand has one or more concealed set of terminal/honor tiles or its equivalent. Recall that one concealed set of terminal/honor tiles is equivalent to \underline{two open} sets of terminal/honor tiles or \underline{two concealed} sets of simple tiles in minipoints. 
Therefore, what we check is whether a hand has any of the following:
\bi \itemsep.2em
\i at least \underline{one concealed} set of terminal/honor tiles;
\i at least \underline{two open} sets of terminal/honor tiles;
\i \underline{one open} set of terminal/honor tiles \emph{and} 
\underline{one concealed} set of simples;
\i at least \underline{two concealed} sets of simples.
\ei
When none of the four above exists in a non-{\jap toitoi} hand without quads, a hand is \emph{always} 40 minipoints if closed and \emph{always} 30 minipoints if open. Therefore, there is really no need of an actual calculation of minipoints with hands like the following. 
\bp
\wan{1}\wan{1}\tong{4}\tong{5}\suo{2}\suo{3}\suo{4}\suo{7}\suo{8}\suo{9}~\zhong\rzhong\zhong\\
\wan{4}\wan{5}\suo{1}\suo{1}\suo{3}\suo{4}\suo{5}~\rdong\dong\dong~\wan{3}\wan{3}\rwan{3}
\wan{2}\wan{2}\wan{2}\wan{7}\wan{8}\suo{2}\suo{3}\suo{4}\suo{6}\suo{6}\suo{7}\suo{8}\suo{9}
\ep
With these hands, we can simply check if it is a closed hand or not to determine if each hand has 40 minipoints (closed) or 30 minipoints (open). 

\bigskip
On the other hand, if a hand satisfies any of the four conditions above, we need to calculate the minipoints by actually counting and summing all the minipoint contributions from all the melds, the head, and the wait to determine the minipoints of a hand. We will discuss this in Section \ref{sec:scores2}.  

%For example, some of the conditions are met in all of the following hands, and so we need to 
%\bp
%\wan{1}\wan{1}\tong{4}\tong{5}\suo{2}\suo{3}\suo{4}\suo{7}\suo{8}\suo{9}~\zhong\rzhong\zhong\\
%\wan{4}\wan{5}\suo{1}\suo{1}\suo{3}\suo{4}\suo{5}~\rdong\dong\dong~\wan{3}\wan{3}\rwan{3}
%\wan{2}\wan{2}\wan{2}\wan{7}\wan{8}\suo{2}\suo{3}\suo{4}\suo{6}\suo{6}\suo{7}\suo{8}\suo{9}
%\ep

\subsection*{3. Getting the scores}

\bigskip
As Figure \ref{fig:mini} makes clear, the case of 30 minipoints is the most important pattern. We thus start with this pattern. We will then proceed to the cases of 40, 20, and 25 minipoints. 

\subsubsection{\fbox{30 minipoints}}
\noindent You get 30 minipoints when you get:
\bi\itemsep.1pt
\i {\jap pinfu ron} (always);
\i closed hand {\jap tsumo} (almost always); or 
\i open hand {\jap ron} / {\jap tsumo} (almost always).
\ei
Scores for 30 minipoints are summarized in Table \ref{tbl:30mp}. 

\begin{table}[h!]
\centering\captionsetup{font=small}\small
\caption{Scores for 30 minipoints} \label{tbl:30mp}
\begin{tabular}{lrrrr}
\toprule
{\jap Han} & \multicolumn{2}{c}{{\jap Ron}}& \multicolumn{2}{c}{{\jap Tsumo}}\\
&\multicolumn{1}{c}{\footnotesize Non-dealer}&\multicolumn{1}{c}{\footnotesize Dealer}&\multicolumn{1}{c}{\footnotesize Non-dealer}&\multicolumn{1}{c}{\footnotesize Dealer}\\
\midrule
1 & 1000 & 1500  & 300-500 & 500-all\\ [\sep]
2 & 2000 & 2900  & 500-1000 & 1000-all\\ [\sep]
3 & 3900 & 5800  & 1000-2000 & 2000-all\\ [\sep]
4 & 7700 & 11600  & 2000-3900 & 3900-all\\ [\sep]
5+ & \multicolumn{4}{c}{limit hand}\\
\bottomrule
\end{tabular}
\end{table}

\bigskip
Compared with the limit hands table (Table \ref{tbl:limithands}), the regularities in the 30 minipoints table are less precise. For example, scores for dealer are only \emph{roughly} 1.5 times those for non-dealer; {\jap tsumo} scores are sometimes slightly bigger than the corresponding {\jap ron} scores. For example, one-{\jap han tsumo} (300-500) gives you $300+300+500 = 1100$, which is slightly bigger than one-{\jap han ron} (1000). 

\bigskip
Therefore, it would be more efficient if we just memorize these patterns as they are, rather than trying to simplify them. Japanese players tend to memorize Table \ref{tbl:30mp} column-wise, as follows:

\bigskip
\begin{itembox}[c]{Scores for 30 minipoints}
\bi
\i {\jap ron} (non-dealer): 10, 20, 39, 77 
\i {\jap ron} (dealer): 15, 29, 58, 116 
\i {\jap tsumo} (non-dealer): 3-5, 5-10, 10-20, 20-39
\ei
\end{itembox}

\bigskip
\noindent
The benefit of memorizing this score table column-wise is that scores get (roughly) twice as big for an additional {\jap han}. Moreover, if we memorize the {\jap tsumo} scores for non-dealer, we can easily derive those for dealer. 
When I was trying to memorize these, I used to recite these sequences a number of times so that they get beaten into my head. 

\subsubsection{\fbox{40 minipoints}}
\noindent You get 40 minipoints when you get:
\bi\itemsep.1pt
\i non-{\jap pinfu} closed hand {\jap ron} (almost always); or
\i {\jap toitoi} {\jap ron} / {\jap tsumo} (almost always).
\ei
Scores for 40 minipoints are summarized in Table \ref{tbl:40mp}. 

\begin{table}[h!]
\centering \captionsetup{font=small}\small
\caption{Scores for 40 minipoints} \label{tbl:40mp}
\begin{tabular}{lrrrr}
\toprule
{\jap Han} & \multicolumn{2}{c}{{\jap Ron}}& \multicolumn{2}{c}{{\jap Tsumo}}\\
&\multicolumn{1}{c}{\footnotesize Non-dealer}&\multicolumn{1}{c}{\footnotesize Dealer}&\multicolumn{1}{c}{\footnotesize Non-dealer}&\multicolumn{1}{c}{\footnotesize Dealer}\\
\midrule
1 & 1300 & 2000  & 400-700 & 700-all\\ [\sep]
2 & 2600 & 3900  & 700-1300 & 1300-all\\ [\sep]
3 & 5200 & 7700  & 1300-2600 & 2600-all\\ [\sep]
4+ & \multicolumn{4}{c}{limit hand}\\
\bottomrule
\end{tabular}
\end{table}
Just like we did with scores for 30 minipoints, I recommend you memorize this column-wise. 
\bigskip
\begin{itembox}[c]{Scores for 40 minipoints}
\bi
\i {\jap ron} (non-dealer): 13, 26, 52, {\jap mangan} 
\i {\jap ron} (dealer): 20, 39, 77, {\jap mangan}
\i {\jap tsumo} (non-dealer): 4-7, 7-13, 13-26, {\jap mangan}
\ei
\end{itembox}

\bigskip\noindent
The good news is that the {\jap ron} score sequence for non-dealer ---13, 26, 52--- is a geometric progression; $13 \times 2 = 26$, and $26 \times 2 = 52$. Also, the {\jap ron} scores for dealer ---20, 39, 77--- should look familiar to you if you have already memorized the 30 minipoints {\jap ron} scores for non-dealer ---10, 20, 39, 77.

\bigskip
\subsubsection{\fbox{20 minipoints} {\jap (pinfu tsumo)}}
	\index{pinfu@{\jap pinfu}}
\noindent Scores for 20 minipoints are summarized in Table \ref{tbl:20mp}. This table is special in the sense that it does not have the {\jap ron} score component nor the one-{\jap han} row. This is because you get 20 minipoints only when you get {\jap pinfu + tsumo} (hence we have at least two {\jap han}). Even though this is a special case, {\jap pinfu} + {\jap tsumo} is far from a rare occurrence. It is thus important to know how to get the correct scores for this case. 

\begin{table}[h!]
\centering\captionsetup{font=small}\small
\caption{Scores for 20 minipoints} \label{tbl:20mp}
\begin{tabular}{lrr}
\toprule
{\jap Han} & \multicolumn{2}{c}{{\jap Tsumo}}\\
&\multicolumn{1}{c}{\footnotesize Non-dealer}&\multicolumn{1}{c}{\footnotesize Dealer}\\
\midrule
2 & 400-700 & 700-all\\ [\sep]
3 & 700-1300 & 1300-all\\ [\sep]
4 & 1300-2600 & 2600-all\\ [\sep]
5+ & \multicolumn{2}{c}{limit hand}\\
\bottomrule
\end{tabular}
\end{table}

\bigskip
Notice that there is an interesting similarity between the 20 minipoints table (Table \ref{tbl:20mp}) and the 40 minipoints table (Table \ref{tbl:40mp}). The scoring patterns are almost identical, except that the required number of {\jap han} to get a certain score is one {\jap han} smaller for 40 minipoints than for 20 minipoints; that is, to get 400-700, we need 2 {\jap han} with 20 minipoints, whereas we only need 1 {\jap han} with 40 minipoints. 

\bigskip
This is not a coincidence. In general, when we double the minipoints, we need one less {\jap han} to get the same score. For example, a 3 {\jap han}--30 minipoints hand and a 2 {\jap han}--60 minipoints hand have the same score (3900 for non-dealer; 5800 for dealer); a 2 {\jap han}--25 minipoints hand and a 1 {\jap han}--50 minipoints hand have the same score (1600 for non-dealer; 2400 for dealer). Therefore, once you memorize scores for 30 minipoints, we can easily deduce scores for hands with 60 minipoints. Similarly, we can deduce scores for hands with 50 or 80 minipoints once we memorize scores for 25 or 40 minipoints, respectively.

\bigskip
\subsubsection{\fbox{25 minipoints} {\jap (chiitoitsu)}}
	\index{chiitoitsu@{\jap chiitoitsu} (Seven Pairs)}
\noindent The final case we cover as part of the basic scoring is scores for a {\jap chiitoitsu} hand. As {\jap chiitoitsu} is a rather exceptional {\jap yaku}, it is given exceptional minipoints --- 25 minipoints. 
Scores for 25 minipoints are summarized in Table \ref{tbl:25mp}. 

\begin{table}[h!]
\centering\captionsetup{font=small}\small
\caption{Scores for 25 minipoints} \label{tbl:25mp}
\begin{tabular}{lrrrr}
\toprule
{\jap Han} & \multicolumn{2}{c}{{\jap Ron}}& \multicolumn{2}{c}{{\jap Tsumo}}\\
&\multicolumn{1}{c}{\footnotesize Non-dealer}&\multicolumn{1}{c}{\footnotesize Dealer}&\multicolumn{1}{c}{\footnotesize Non-dealer}&\multicolumn{1}{c}{\footnotesize Dealer}\\
\midrule
2 & 1600 & 2400  & \\ [\sep]
3 & 3200 & 4800  & 800-1600 & 1600-all\\ [\sep]
4 & 6400 & 9600  & 1600-3200 & 3200-all\\ [\sep]
5+ & \multicolumn{4}{c}{limit hand}\\
\bottomrule
\end{tabular}
\end{table}
Since {\jap chiitoitsu} is itself a two-{\jap han} {\jap yaku}, the scoring table only starts with two {\jap han} for {\jap ron} and three {\jap han} for {\jap tsumo}. 
Again, we memorize this table column-wise. 

\bigskip
\begin{itembox}[c]{Scores for 25 minipoints}
\bi
\i {\jap ron} (non-dealer): 16, 32, 64, {\jap mangan}
\i {\jap ron} (dealer): 24, 48, 96, {\jap mangan}
\i {\jap tsumo} (non-dealer): 8-16, 16-32, {\jap mangan}
\ei
\end{itembox}

\bigskip
All the sequences are a geometric progression, making it relatively easy to memorize. 

\bigskip
\subsubsection{Practice, practice, practice}
This completes the basic methods of score calculation. Memorizing all the scores for limit hands as well as the cases of 30, 40, 20, and 25 minipoints should be more than enough. 
Of course, no one will be able to master this method just by reading and understanding the materials in this chapter. You would need to actually practice what you have learned, and you will have to do so repeatedly.

\newpage
\begin{boxnote} \small
{\bf\normalsize Notes on {\jap pinfu tsumo}}
\index{pinfu@{\jap pinfu}}
\bigskip

One of the common mistakes that beginners tend to make is to claim 1000-2000 for riichi + {\jap pinfu + tsumo} (mistaking 3 {\jap han}--20 minipoints for 3 {\jap han}--30 minipoints) or claim {\jap mangan} for riichi + {\jap pinfu + tsumo + dora} (mistaking 4 {\jap han}--20 minipoints for 4 {\jap han}--40 minipoints). 
If you have trouble wrapping your head around why {\jap pinfu + tsumo} hands are given lower minipoints, knowing the origin of this rule might be helpful. 
\bigskip

The {\jap yaku} {\jap pinfu} is realized when a hand has no component that generates an additional minipoint. A {\jap pinfu} hand cannot have a set, edge wait, closed wait, single wait, or a pair of value tiles because they all generate an additional minipoint. Recall that {\jap tsumo} also generates 2 minipoints. Therefore, logically speaking, {\jap pinfu} cannot be claimed when you win it by {\jap tsumo}. In fact, some traditional mahjong rule sets do not allow a combination of {\jap pinfu} and {\jap tsumo}. 
Under such rule sets, when you win a {\jap pinfu}-only hand by {\jap tsumo}, you are allowed to claim {\jap tsumo} only, giving you 1 {\jap han}--30 minipoints (20 base minipoints + 2 minipoints for {\jap tsumo} = 22, rounded up to 30 minipoints), which gives you 300-500.  

%\hfill (\textit{continued on next page})
%
%\end{boxnote}
%
%\newpage
%\begin{boxnote} \small
\bigskip
However, some people thought that this is a bit unfair, claiming that a {\jap pinfu tsumo} hand should be given a higher score than a {\jap tsumo}-only hand. At the same time, they recognized that the score for {\jap pinfu + tsumo} should be lower than that for a ``proper'' 2-{\jap han} {\jap tsumo} hand (e.g., {\jap tanyao + tsumo}). 
Therefore, they decided that {\jap pinfu + tsumo} should be placed in between these two --- 1 {\jap han}--30 minipoints ({\jap tsumo} only; 300-500) and 2 {\jap han}--30 minipoints ({\jap tanyao + tsumo}; 500-1000) --- giving it the score of 400-700. Therefore, the score for riichi + {\jap pinfu + tsumo} (3 {\jap han}--20 minipoints; 700-1300) is higher than riichi + {\jap tsumo} (2 {\jap han}--30 minipoints; 500-1000) but lower than riichi + {\jap tanyao + tsumo} (3 {\jap han}--30 minipoints; 1000-2000). 
\end{boxnote}

\newpage

\section{Advanced scoring} \label{sec:scores2}

As I mentioned in the previous section, a hand with one or more concealed set of terminal/honor tiles (or its equivalent) or quads may have unusually high minipoints, calling for an actual calculation of minipoints. 

\subsection{Minipoint calculation}
	\index{fu@{\jap fu} (minipoint)} \index{minipoint ({\jap fu})}
Let's first review the basics of minipoint calculations. 
All standard hands (i.e., hands with melds) have the base 20 minipoints. Then, we add the following minipoints depending on how we win the hand:
\bi \itemsep0em
\i {\jap tsumo} (open or closed, except for {\jap pinfu}): 2
\i {\jap ron} (closed): 10
\i {\jap ron} (open): 0
\ei 
We add further minipoints for each set and quad in a hand depending on whether it is a concealed one or an open one. Table \ref{tbl:minip_set} summarizes minipoint contributions from a set and a quad. 
{\begin{table}[h!]\centering\small\captionsetup{font=small}
\caption{Minipoint contributions from a set and a quad} \label{tbl:minip_set}
\begin{tabular}{l r c c}
\toprule
& \multicolumn{1}{c}{Tile} & \multicolumn{2}{c}{Minipoint}\\
&  & {\footnotesize Open} & {\footnotesize Concealed}\\
\midrule
set	& simple & 2 & 4\\
	& terminal/honor & 4 & 8\\
\midrule
quad	 & simple & 8 & 16\\
	& terminal/honor & 16 & 32\\
\bottomrule
\end{tabular}
\end{table}}

\newpage
\noindent Finally, we add 2 minipoints for each of the following, if any. 
\bi \itemsep0.1em
\i Pair of dragon tiles
\i Pair of seat wind tiles
\i Pair of prevailing wind tiles
\i Closed, edge, or single wait
\ei

When the head of a hand is of seat wind \emph{and} prevailing wind (e.g., {\LARGE\dong} for the East player in the East round), we get 2 + 2 = 4 minipoints.\footnote{This is the case with EMA rules and {\jap Tenhou} rules, but this is not a universally adopted rule.} 
If the wait is either side wait or dual {\jap pon} wait, we don't get any minipoint for it.
As we saw when we discussed wait patterns in \ref{sec:waits}, we may get different minipoints depending on which of the multiple winning tiles to win on. 
For example, consider the following hand.
\bp
\wan{3}\wan{4}\wan{4}\wan{5}\wan{6}\tong{4}\tong{4}\zhong\zhong\zhong~\suo{3}\suo{3}\rsuo{3}
\ep
The hand is waiting for {\LARGE\wan{2}-\wan{5}}. If we win by {\jap ron} on {\LARGE\wan{2}}, we get no minipoint for the wait and so this hand has 30 minipoints (base 20 + 8 for a concealed set of honors + 2 for an open set of simple = 30). However, if we win this hand on {\LARGE\wan{5}}, we get additional 2 minipoints for closed wait. This is because {\LARGE\wan{3}\wan{4}\wan{4}\wan{5}\wan{6}} can be thought of as {\LARGE\wan{3}\wan{4}\wan{5} + \wan{4}\wan{6}}. Therefore, the hand has 40 minipoints in that case (30 + 2 = 32, rounded up to 40). 

\bigskip
\subsection{Scores for 50 minipoints or above}
\bigskip
When a hand has one or more concealed set of honor tiles, the hand may have 50 minipoints or above. You may want to memorize the case of 50 minipoints, summarized below. 
If you are a perfectionist, you may also want to memorize the cases of 70 and 110 minipoints as well, but I can assure you that it would not be worth the effort. 

\subsubsection{\fbox{50 minipoints}}
\noindent Scores for 50 minipoints are quite easy to memorize if you have already memorized scores for 25 minipoints ({\jap chiitoitsu}), summarized in \ref{tbl:25mp}. Recall that the score for a 1 {\jap han}--50 minipoints hand should be the same as that for a 2 {\jap han}--25 minipoints hand.

\begin{table}[h!]
\centering\captionsetup{font=small}\small
\caption{Scores for 50 minipoints} \label{tbl:50mp}
\begin{tabular}{lrrrr}
\toprule
{\jap Han} & \multicolumn{2}{c}{{\jap Ron}}& \multicolumn{2}{c}{{\jap Tsumo}}\\
&\multicolumn{1}{c}{\footnotesize Non-dealer}&\multicolumn{1}{c}{\footnotesize Dealer}&\multicolumn{1}{c}{\footnotesize Non-dealer}&\multicolumn{1}{c}{\footnotesize Dealer}\\
\midrule
1 & 1600 & 2400  & 400-800 & 800-all\\ [\sep]
2 & 3200 & 4800  & 800-1600 & 1600-all\\ [\sep]
3 & 6400 & 9600  & 1600-3200 & 3200-all\\ [\sep]
4+ & \multicolumn{4}{c}{limit hand}\\
\bottomrule
\end{tabular}
\end{table}

\subsubsection{\fbox{70 minipoints}}
\noindent Hands with 70 minipoints do not appear very often (probably once in 20 games or so). Table \ref{tbl:70mp} summarizes scores for 70 minipoints.
If you would like to memorize the table, notice that it is sequential (until the end): 23 (non-delaer) $\rightarrow$ 34 (dealer) $\rightarrow$ 45 (non-dealer) $\rightarrow$ 68 (dealer).
\begin{table}[h!]
\centering\captionsetup{font=small}\small
\caption{Scores for 70 minipoints} \label{tbl:70mp}
\begin{tabular}{lrrrr}
\toprule
{\jap Han} & \multicolumn{2}{c}{{\jap Ron}}& \multicolumn{2}{c}{{\jap Tsumo}}\\
&\multicolumn{1}{c}{\footnotesize Non-dealer}&\multicolumn{1}{c}{\footnotesize Dealer}&\multicolumn{1}{c}{\footnotesize Non-dealer}&\multicolumn{1}{c}{\footnotesize Dealer}\\
\midrule
1 & 2300 & 3400  & 600-1200 & 1200-all\\ [\sep]
2 & 4500 & 6800  & 1200-2300 & 2300-all\\ [\sep]
3+ & \multicolumn{4}{c}{limit hand}\\
\bottomrule
\end{tabular}
\end{table}

\subsubsection{\fbox{110 minipoints}}
\noindent For the sake of completeness, Table \ref{tbl:110mp} summarizes scores for 110 minipoints. 

\begin{table}[h!]
\centering\captionsetup{font=small}\small
\caption{Scores for 110 minipoints} \label{tbl:110mp}
\begin{tabular}{lrrrr}
\toprule
{\jap Han} & \multicolumn{2}{c}{{\jap Ron}}& \multicolumn{2}{c}{{\jap Tsumo}}\\
&\multicolumn{1}{c}{\footnotesize Non-dealer}&\multicolumn{1}{c}{\footnotesize Dealer}&\multicolumn{1}{c}{\footnotesize Non-dealer}&\multicolumn{1}{c}{\footnotesize Dealer}\\
\midrule
1 & 3600 & 5300  & --- & ---\\ [\sep]
2 & 7100 & 10600  & 1800-3600 & 3600-all\\ [\sep]
3+ & \multicolumn{4}{c}{limit hand}\\
\bottomrule
\end{tabular}
\end{table}

\newpage
110 minipoints occur only in extremely rare occasions. Consider the following hand. 
\bp
\tong{6}\tong{7}\tong{8}\dong\dong\zhong\zhong~\suo{30}\suo{1}\suo{1}\suo{30}~~\suo{30}\suo{9}\suo{9}\suo{30}
\ep
Suppose you are in the East round. If the dealer wins this hand by {\jap ron} on {\LARGE\zhong}, he gets 20 (base) + 10 (closed hand {\jap ron}) + 4 (pair of seat \& prevailing wind tiles) + 4 (open set of honors) + 32 (concealed quad of terminals) + 32 (concealed quad of terminals) = 102, rounded up to 110 minipoints. The score for 1 {\jap han}--110 minipoints is 5300. 

\bigskip
If he wins by {\jap ron} on {\LARGE\dong}, on the other hand, he gets more {\jap han} (set of seat \& prevailing wind) but lower minipoints. This is because the minipoint contribution of the pair of {\LARGE\zhong} (2) is smaller than that of the pair of {\LARGE\dong} (4). Since the score of 2 {\jap han}--100 minipoints hand is the same as that of 3 {\jap han}--50 minipoints hand, he obtains 9600 points.

\newpage
\subsection{Examples}

Let's see how scores change as we advance a hand. For each of the examples below, try calculating the scores for different winning tiles and for {\jap tsumo} and {\jap ron}. 

\begin{itembox}[r]{Scoring 1}
\bp
\wan{2}\wan{2}\wan{3}\wan{4}\wan{5}\tong{4}\tong{5}\tong{5}\tong{6}\tong{7}\zhong\zhong\zhong
\ep
\vspace{-10pt} What are the scores?
\end{itembox}

\bigskip
\noindent If you win this hand by {\jap ron}, the hand has 1 {\jap han} (Red Dragon) and 40 minipoints: 20 (base) + 10 (closed hand {\jap ron}) + 8 (concealed set of honors) = 38, rounded up to 40, so you get 1300 points. 

\bigskip
If you win it by drawing {\LARGE\tong{3}}, the hand has an additional {\jap yaku}, {\jap menzen tsumo} (Fully Concealed Hand), with 30 minipoints: 20 + 8 + 2 ({\jap tsumo}) = 30. You thus get 500-1000 {\jap tsumo} = 2000 points. 

\bigskip
However, if you win it by drawing {\LARGE\tong{6}}, you get 40 minipoints because of the additional 2 minipoints for \\closed wait: {\LARGE\tong{5}\tong{7}}. You thus get 700-1300 {\jap tsumo} = 2700 points.

\bigskip

Let's say you draw {\LARGE\tong{5}}. What would you discard?
\bp
\wan{2}\wan{2}\wan{3}\wan{4}\wan{5}\tong{4}\tong{5}\tong{5}\tong{6}\tong{7}\zhong\zhong\zhong~\tong{5}\\
\hfill\footnotesize{Draw~~~~~~~~~~~~~~~}
\ep
If you discard {\LARGE\tong{5}}, the wait is {\LARGE\tong{3}-\tong{6}} (2 kinds--7 tiles). If you discard {\LARGE\tong{4}}, however, you get a 3-way wait: {\LARGE\tong{5}-\tong{8} \wan{2}} (3 kinds--7 tiles). Let's say you choose the latter, resulting in the following hand. Now, let's think about the scores for each winning tile. 

\begin{itembox}[r]{Scoring 2}
\bp
\wan{2}\wan{2}\wan{3}\wan{4}\wan{5}\tong{5}\tong{5}\tong{5}\tong{6}\tong{7}\zhong\zhong\zhong\\
\ep
\vspace{-10pt} What are the scores?
\end{itembox}

\bigskip
\noindent If you win this hand on {\LARGE\wan{2}}, the hand is still 1 {\jap han}--40 minipoints = 1300. However, if you win on {\LARGE\tong{5}} or {\LARGE\tong{8}}, the three tiles of {\LARGE\tong{5}} within the hand are treated as a concealed set, giving you 4 additional minipoints: 20 (base) + 10 (closed {\jap ron}) + 8 (set of \zhong) + 4 (set of \tong{5}) = 42, rounded up to 50 minipoints. You thus get 1 {\jap han}--50 minipoints = 1600 points. If you win by {\jap tsumo}, you get 40 minipoints so you will get 700-1300 = 2700 points. 

\bigskip

Let's say you draw {\LARGE\wan{3}}. What would you discard? 
\bp
\wan{2}\wan{2}\wan{3}\wan{4}\wan{5}\tong{5}\tong{5}\tong{5}\tong{6}\tong{7}\zhong\zhong\zhong~\wan{3}\\
\hfill\footnotesize{Draw~~~~~~~~~~~~~~~}
\ep
If you discard {\LARGE\wan{5}}, the wait is {\LARGE\wan{1}-\wan{4}}. Let's think about the scores for each winning tile. 
\begin{itembox}[r]{Scoring 3}
\bp
\wan{2}\wan{2}\wan{3}\wan{3}\wan{4}\tong{5}\tong{5}\tong{5}\tong{6}\tong{7}\zhong\zhong\zhong
\ep
\vspace{-10pt} What are the scores?
\end{itembox}

\bigskip
\noindent If you win the hand by {\jap ron} on {\LARGE\wan{4}}, you get an additional {\jap yaku}, {\jap iipeiko} (Pure Double Chow), giving you 2 {\jap han}--40 minipoints = 2600 points. 

\bigskip
If you win the hand by {\jap tsumo}, the minipoints are now lower than before because you have side wait and only one concealed set; we cannot think of the three tiles of {\LARGE\tong{5}} as a set any more. You get 2 {\jap han}--30 minipoints if you draw {\LARGE\wan{1}} (500-1000 {\jap tsumo} = 2000), whereas you get 3 {\jap han}--30 minipoints if you draw {\LARGE\wan{4}} (1000-2000 {\jap tsumo} = 4000).

\bigskip

Let's say you draw {\LARGE\tong{6}}. What would you discard? 
\bp
\wan{2}\wan{2}\wan{3}\wan{3}\wan{4}\tong{5}\tong{5}\tong{5}\tong{6}\tong{7}\zhong\zhong\zhong~\tong{6}\\
\hfill\footnotesize{Draw~~~~~~~~~~~~~~~}
\ep
Discarding {\LARGE\tong{7}} is the best option. To understand why, let's think about the scores. 

\begin{itembox}[r]{Scoring 4}
\bp
\wan{2}\wan{2}\wan{3}\wan{3}\wan{4}\tong{5}\tong{5}\tong{5}\tong{6}\tong{6}\zhong\zhong\zhong
\ep
\vspace{-10pt} What are the scores?
\end{itembox}

\bigskip
\noindent Notice that the wait and the potential {\jap han} counts did not change at all. However, you get increased minipoints because you now have the concealed set of {\LARGE\tong{5}} back again. You will get 2 {\jap han}--50 minipoints = 3200 points if you win by {\jap ron} on {\LARGE\wan{4}}. If you draw {\LARGE\wan{4}}, you will get 3 {\jap han}--40 minipoints, giving you 1300-2600 {\jap tsumo} = 5200 points. 

\bigskip

Let's say you draw another {\LARGE\tong{6}}. What would you discard? 
\bp
\wan{2}\wan{2}\wan{3}\wan{3}\wan{4}\tong{5}\tong{5}\tong{5}\tong{6}\tong{6}\zhong\zhong\zhong~\tong{6}\\
\hfill\footnotesize{Draw~~~~~~~~~~~~~~~}
\ep
The best discard is {\LARGE\wan{4}}, which makes this hand a {\jap toitoi} (All Pungs) hand, as follows.

\begin{itembox}[r]{Scoring 5}
\bp
\wan{2}\wan{2}\wan{3}\wan{3}\tong{5}\tong{5}\tong{5}\tong{6}\tong{6}\tong{6}\zhong\zhong\zhong
\ep
\vspace{-10pt} What are the scores?
\end{itembox}

\bigskip
\noindent The hand has three concealed sets already, giving you at least {\jap san anko} (Three Concealed Pungs) in addition to {\jap toitoi} and Red Dragon. Now you no longer need any tedious minipoints calculation. If you win this hand by {\jap tsumo}, it is {\jap yakuman} ({\jap su anko}; Four Concealed Pungs). If you win it by {\jap ron}, you get five {\jap han} ({\jap toitoi}, {\jap san anko}, and Red Dragon); it is {\jap mangan} regardless of minipoints. 

\newpage
\section{Scoring tables}

{\begin{table}[h!]\centering\footnotesize\captionsetup{font=footnotesize}
\caption{Scores for non-dealer} \label{tbl:scores1}
\begin{tabular}{l c c c c}
\toprule
Minipoints & 1 {\jap han} & 2 {\jap han} & 3 {\jap han} &4 {\jap han}\\
\midrule
20 & --- & --- & --- & --- \\
& --- & (400-700) & (700-1300) & (1300-2600)\\ [\sep]
25 & --- & 1600 & 3200 & 6400\\
& --- & --- & (800-1600) & (1600-3200)\\ [\sep]
30 & 1000 & 2000 & 3900 & 7700\\
& (300-500) & (500-1000) & (1000-2000) & (2000-3900)\\ [\sep]
40 & 1300 & 2600 & 5200 & 8000\\
& (400-700) & (700-1300) & (1300-2600) & (2000-4000)\\ [\sep]
50 & 1600 & 3200 & 6400 & 8000\\
& (400-800) & (800-1600) & (1600-3200) & (2000-4000)\\ [\sep]
60 & 2000 & 3900 & 7700 & 8000\\
& (500-1000) & (1000-2000) & (2000-3900)& (2000-4000)\\ [\sep]
70 & 2300 & 4500 & 8000 & 8000\\
& (600-1200) & (1200-2300) & (2000-4000)& (2000-4000)\\ [\sep]
80 & 2600 & 5200 & 8000 & 8000\\
& (700-1300) & (1300-2600) & (2000-4000)& (2000-4000)\\ [\sep]
90 & 2900 & 5800 & 8000 & 8000\\
& (800-1500) & (1500-2900) & (2000-4000)& (2000-4000)\\ [\sep]
100 & 3200 & 6400 & 8000 & 8000\\
& (800-1600) & (1600-3200) & (2000-4000)& (2000-4000)\\ [\sep]
110 & 3600 & 7100 & 8000 & 8000\\
& --- & (1800-3600) & (2000-4000)& (2000-4000)\\ [\sep]
\bottomrule
\end{tabular}\\
{\vsps \textit{Note:} Numbers in parentheses are {\jap tsumo} scores.}
\end{table}}

{\begin{table}[h!]\centering
\small\captionsetup{font=small}
\caption{Scores for dealer} \label{tbl:scores2}
\begin{tabular}{l c c c c}
\toprule
Minipoints & 1 {\jap han} & 2 {\jap han} & 3 {\jap han} &4 {\jap han}\\
\midrule
20 & --- & --- & --- & --- \\
& --- & (700) & (1300) & (2600)\\ [\sep]
25 & --- & 2400 & 4800 & 9600\\
& --- & --- & (1600) & (3200)\\ [\sep]
30 & 1500 & 2900 & 5800 & 11600\\
& (500) & (1000) & (2000) & (3900)\\ [\sep]
40 & 2000 & 3900 & 7700 & 12000\\
& (700) & (1300) & (2600) & (4000)\\ [\sep]
50 & 2400 & 4800 & 9600 & 12000\\
& (800) & (1600) & (3200) & (4000)\\ [\sep]
60 & 2900 & 5800 & 11600 & 12000\\
& (1000) & (2000) & (3900)& (4000)\\ [\sep]
70 & 3400 & 6800 & 12000 & 12000\\
& (1200) & (2300) & (4000)& (4000)\\ [\sep]
80 & 3900 & 7700 & 12000 & 12000\\
& (1200) & (2300) & (4000)& (4000)\\ [\sep]
90 & 4400 & 8700 & 12000 & 12000\\
& (1500) & (2900) & (4000)& (4000)\\ [\sep]
100 & 4800 & 9600 & 12000 & 12000\\
& (1600) & (3200) & (4000)& (4000)\\ [\sep]
110 & 5300 & 10600 & 12000 & 12000\\
& --- & (3600) & (4000)& (4000)\\ [\sep]
\bottomrule
\end{tabular}\\
{\vsps \textit{Note:} Numbers in parentheses are {\jap tsumo} scores.}
\end{table}}






