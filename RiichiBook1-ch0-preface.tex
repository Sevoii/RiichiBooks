%~~~~~~~~~~~~~~~~~~~~~~~~~~~~~~~~~~~~~~~~~~~~~~~~~
% Riichi Book 1, Preface
%~~~~~~~~~~~~~~~~~~~~~~~~~~~~~~~~~~~~~~~~~~~~~~~~~
\chapter{Preface}
\thispagestyle{empty}

When I started learning Mahong in 2024, I was pleasantly surprised that there was such a large and welcoming Riichi Mahjong community.
In the past few months, I've been playing regularly with friends, so I've decided to make this book (along with the help of Shelly) for the community :)

Thank you to Diana for providing the source material and serving as a foundation for my own Mahjong journey.

\vfill

%When I moved to England in 2013, I was pleasantly surprised to learn that riichi mahjong (modern Japanese mahjong) is quite popular in Europe. In the past two years, I have had the pleasure of playing riichi in London, Guildford, Kent, Oxford, Aachen, Copenhagen,\\ Prague, and Vienna, along with players from Austria, China, Czech Republic, Denmark, Estonia, Finland, France, Germany, Italy, Japan, the Netherlands, Poland, Russia, Slovakia, Sweden, the UK, and the United States.
%
%\bigskip
%European players have been remarkably successful in organizing tournaments open to anyone who plays the game. These tournaments --- held at least once a month somewhere in Europe --- are run by local mahjong players in each country under the auspices of the European Mahjong Association (EMA).\footnote{\url{http://mahjong-europe.org/}} \index{european@EMA}
%Founded in 2005, EMA has been doing a fantastic job in maintaining common rule sets,%
%\footnote{EMA's official rule book for riichi mahjong is available online at \url{http://mahjong-europe.org/portal/images/docs/Riichi-rules-2016-EN.pdf} (last revised in 2016). At the time of writing this book, EMA is in the process of revising the rule book.
%Explanations of EMA rules in this book are based on the revised rules. New rules will come into effect from April, 2016.
%}
%keeping a player ranking system, and doing many other useful things to promote the playing of mahjong across Europe.
%
%\bigskip
%Although I have come across a few good players in Europe, I came to realize that a lot of players here are not very well-versed in the basic principles of competitive mahjong strategies. Of course, playing competitively is not the only way to enjoy the game.
%I am also not claiming that I know the magic formula to win because there is no such thing. Nevertheless, there is a set of basic principles worth learning for any aspiring players. I believe the level of sophistication among European players could be much improved if these principles are more widely shared. Unfortunately, however, learning resources currently available for non-Japanese audience are somewhat limited.\footnote{There are already a few English books for beginners. There are also several excellent blog posts on technical details about mahjong strategies. However, there appears to be a huge gap between these two sets of resources. Introductory books do not cover strategies extensively, whereas blog posts tend to be too advanced even for intermediate players.}
%
%\bigskip
%I have thus decided to write a book on riichi mahjong strategies for European players, primarily with beginners and intermediate players in mind. I then ended up splitting the book into two volumes; Book I is intended for beginners and intermediate players ({\jap Tenhou} rank of 四段 or below), while Book II is meant for more advanced players. The two books are \emph{not} intended for complete novices who do not know how to play riichi mahjong.\footnote{If you want to learn how to play riichi, I'd recommend Barr (2009).} The target reader is anyone who has played riichi mahjong before and wants to improve their skills further.
%\index{Barr@Barr, Jenn}
%
%\bigskip
%I have three main goals in preparing these books. First, I will introduce a set of English terminology of riichi mahjong.
%``In beginning was Word,'' scripture tells us. Knowing the names of particular tile combinations, situations, and strategies will allow us to be conscious of them and to be able to talk about them with our fellow players after the game.
%
%\bigskip
%My second goal is to introduce the principles of tile efficiency.
%Book I and Book II both cover tile efficiency, but at different levels. Book I offers an introduction to tile efficiency, covering very basic mechanisms only. I plan to cover more advanced materials in Book II.
%My third goal is to introduce a set of simple strategies regarding critical judgements such as whether or not to call {\jap riichi}, whether to push or to fold, and whether or not to meld.
%
%\bigskip
%A lot of the materials covered in the books were introduced to me through the writings of a notable Japanese mahjong player and manga author, Masa\-yuki Kata\-yama. Mr.\textasciitildeKata\-yama is an accomplished riichi player and arguably the best mahjong manga author in the world. Some of the strategies introduced here are unabashedly stolen from Mr. Kata\-yama's masterpiece manga storybook $Uta\-hime$ $Obaka\-miiko$ (『打姫\-オバカ\-ミーコ』).
%I strongly encourage you to read it yourself if you read Japanese, although I realize that you would not be reading my book if you understood Japanese.
%\index{Katayama@Katayama, Masayuki}
%
%\bigskip
%Another Japanese author whose work has been influential in the writing of Book I is Makoto Fukuchi. Mr.\textasciitildeFukuchi is also a distinguished riichi player and the best-selling author of mahjong strategy books. A part of the exposition of the five-block method in Chapter \ref{ch:five}\textasciitildeis based on Mr.\textasciitildeFukuchi's skillful explanation in his books. \index{Fukuchi@Fukuchi, Makoto}
%
%\bigskip
%I am also indebted to a lot of friends I have become acquainted with through mahjong in Europe. Philipp Martin has read an early draft of the book and provided me with valuable comments and encouragement. I am also grateful to Gemma Sakamoto, who has been hosting a monthly mahjong get-together in London.
%Finally, my thanks go to Ian Fraser, one of the founders of the UK Mahjong Association.
%Without the efforts of Ian and his team, I would not have been able to get to know so many fellow players in the UK and in Europe.
%
%\bigskip
%The cover photo (\copyright\textasciitildeKatar\'{i}na M\'{o}zov\'{a}) is from the 2015 Bratislava Riichi Open Tournament. I thank Katar\'{i}na and Riichi Mahjong Slovakia (especially Matej Laba\v{s}) for giving me their permission to use it.
%
%\bigskip
%After I made the book publicly available in January 2016, a lot of people have given me feedback on various aspects of the book. Based on their feedback, I corrected some terminology inconsistencies and typos. In particular, I thank David Clarke, Aaron Ebejer, Nicolas Giaconia, Grant Mahoney, Ting, Chris Rowe, Mike Liang, and Max Lu for their valuable inputs.
%
%\vfill
%
%\hfill Daina Chiba\\
%\hfill London, UK\\
%\hfill January 10, 2016\\
%\hfill (updated on \today)

\section*{Plan of da book, nya\textasciitilde!}

I play by feel, but since you aren’t as good as me, you will need to learn
some techniques to match my skill! If you wanna get super-duper good at
paw-jong... oops, I mean \textit{Mahjong}, you gotta learn the theories, nya\textasciitilde!
But, um, also do lots of practice stuff too, nya? Like playin' games... um,
lots of 'em, nya! Preferably wif paw-layers who are like, um, stronger than you
are, nya! B-but, uh, before the whole, um, online Mahjong thingies, it was
suuuuper hard if you didn’t live in, um, Japan, nya! (\textit{fluffs tail nervously})

But, um, nyaaa\textasciitilde! Now there are these cool online Mahjong places where you
can, um, play like a bajillion games (or more, nyaaa\textasciitilde!), all while sittin'
comfy at home, but the best is Mahjong Soul! You can find other players like me
anytime, like 24/7, nya\textasciitilde! Oh, and they keep track of your games so you can, um,
re-watch yourself bein' awesome—or makin’ silly mistakes, nyaaa\textasciitilde! (\textit{bonks
head} Whoopsies\textasciitilde!)

S-so, nya, you should practice your Mahjong skills online while also, um,
readin’ this book-thingy! You don’t hafta, um, finish it all before playin’,
nya! Just jump in, nyaaa\textasciitilde! Play a bunch of games, then, um, come back and read
the paw-ts of the book you need, nya\textasciitilde! (\textit{wiggles ears excitedly})

This book is, um, split into four paw-ts, nya! Paw-t I is, um, an
intro-nyaduction (that’s a big word, nya\textasciitilde!) to you to play online!
(Of course we go onto Mahjong Soul nya\textasciitilde!)

Paw-ts II and III are like, um, the juicy meat of the book, nya\textasciitilde! (Mmm,
meat… nyaaa\textasciitilde!) Paw-t II talks about basic tile efficiency—um, stuff that makes
your hand all fast and strong, nya! There’s basic terms in Chapter 3, and then,
um, the five-block thingy in Chapter 4, nya\textasciitilde! Oh, and some tips for makin’ yaku
in Chapter 5, nya\textasciitilde! Paw-t III has big brain strats like score
calcu…calcu-whatevering (Chapter 6), riichi thingies (Chapter 7), defensey
stuff (Chapter 8), and, um, how to meld and win games, nya\textasciitilde! (Chapter 9 and 10,
nya\textasciitilde!)

Oh, and, um, there’s some appendixes (is that the right word, nya?) about
offline Mahjong manners (Chapter A, nya\textasciitilde!) and more books to read (Chapter B,
nyaaa\textasciitilde!)

If you click on the numbers and letters in the book, nya, it’ll take you
right to the page you need, nya\textasciitilde! (Magic, nya\textasciitilde!)

Whew, that was a lot, nya\textasciitilde! (\textit{flops down, purring})




