%~~~~~~~~~~~~~~~~~~~~~~~~~~~~~~~~~~~~~~~~~~~~~~~~~
% Riichi Book 1, Chapter 12: EMA
%~~~~~~~~~~~~~~~~~~~~~~~~~~~~~~~~~~~~~~~~~~~~~~~~~

\chapter{Playing at EMA tournaments}\label{ch:ema}

%
%I view EMA tournaments to be primarily a social event; they are a great opportunity to get to know our fellow players from other countries, learn from them, perhaps teach them something, and, most of all, to have fun playing with them. 
%Of course, I take mahjong very seriously and always play to win, but I have also come to realize that tournaments are probably not the best place to test your mahjong skills. 
%Despite the best intentions of tournament organizers, it is often impossible for them to give all participants a fair and level playing field. 
%Some players are sometimes painfully slow for you; some players may be probably too fast for you. 
%
%If you really, really want to win a tournament, there are a few cunning strategies available for you. These tactics are certainly unethical and you'll probably get hated by your fellow players, but they are currently not outlawed by the EMA rulebook.\footnote{This chapter is meant to be a joke; I hope that's clear from my writing.}
%
%
%\section{Stall As Much As Possible}
%Mahjong is a game of chance although skills also play an important part. 
%Strong players will probably win against unskilled opponents if they play 100 games or so, but they can easily lose to a novice if they get to play only a few hands. 
%Therefore, in order for a weak player like yourself to increase the chance of winning against strong opponents, you should slow down the game as much as possible.\footnote{I just called you a weak player because a strong player would not want to adopt any of the tactics described here.} 
%At EMA tournaments, games are timed (75 minutes and 2 more hands) but actions are NOT timed. Take advantage of this by making your every single action (e.g., drawing, discarding, counting, shuffling, etc.) as slow as humanly possible. 
%
%You are allowed to take as much as 3 seconds (!) to call a pung. Do it. 
%
%Especially when you are ahead in the game, you really need to stall. For example, if you get lucky and win a mangan hand in East 1, start a serious stalling. Don't let the game proceed into the South round.
%
%\begin{itemize}
%\item Take as much time as possible in building the wall. 
%\item In taking the tiles to start a hand, do not look at the tiles as you take them; always wait until you have all the 13 (14 if you are a dealer) tiles in front of you, then slowly arrange the tiles upright, have a look at them, and start sorting them. That way, you have a good excuse to be slow in discarding the first tile. 
%\item Whenever someone discards a dora or red tiles, or any tiles for that matter, say ``Wait!'' and stop the game, pretend that you are thinking about pung, chow, or ron. Look at your hand, look at the discarded tile, then look at your hand again. Do it a few times before you let it go. 
%\item Go to toilet during the game at least once. 
%\item If you are keeping the scoresheet, make as many miscalculation as possible. Constantly recalculate and correct the scores. You may want to just stare at the score sheet after every single hand, pretending to think that there is something wrong. Of course, you need to volunteer to be the score keeper. 
%\end{itemize}
%
%If you don't want to be seen as employing a stall tactics, learn to play fast. 
%
%\section{Call a Pung After Anyone Tsumos}
%
%This tactics can be employed when the Left player or the Facing player declares a tsumo. When that happens, in a split second, you need to declare ``Pung!'' at the tile just discarded by the player seated on the left of the winning player. 
%For example, if the Left player tsumos, claim pung at the tile discarded by the Facing player. As long as you call a pung within three seconds (which is a $really$ long time) after the previous discard, you are entitled to claim the discarded tile.\footnote{In my opinion, this three-second rule should be abolished, which would render this and the next tactics impossible.}
%
%The great value of tactics is that it doesn't really matter if you don't have the correct pung materials (pair of tiles same as the discarded one). You can still legally claim a pung, take the discarded tile, place it on your right along with any arbitrary tiles (they'd better be of the same suit, but it doesn't matter) from your hand, and discard one tile. 
%Of course, you will get a dead hand penalty if it's not the correct pung, but that is nothing compared with your opponent winning a hand by tsumo! Keep in mind, though, that you'll get a chombo penalty if you call pung when you had already got a dead hand before. 
%Another nontrivial benefit of this tactics is that the winning tile that the sucker declared a tsumo with has probably been shown to you and other players by the time you call a pung. 
%
%
%\section{Call a Pung After Anyone Chows}
%
%When your opponent chows, you have 3 seconds to declare pung according to EMA rules. I find this rule absurd. Chow should take precedent if the call has been made prior to a pung call. Pung / kong / ron should take precedent ONLY if calls are made simultaneously. 
%Giving as much as 3 seconds to a late caller is tantamount to encouraging stall tactics and punishing fast players. 
%
%
%Anyhow, as long as this rule remains in place, why not take advantage of it? 
%When your opponent chows, declare pung. Again, it doesn't matter if you don't have the right pung materials. You'll get a dead hand but you can block your opponent growing a big hand. 
%
%\section{Defense against the dark arts: Proposed amendments to EMA rules}
%



%\chapter{How to Become a Respected Player}
%
%\section{Don't win a ``crap'' hand}
%Crap hand is totally different from cheap hand. For example, if you win a 300--500 hand when the dealer riiched it's cheap but meaningful. 
%However, having four pungs to win a 1000 or 2000 hand in East-1 has of little value. 
%Likewise, if you are behind other players in South-4 you should at least try to improve your position. 
%That is, don't win a hand that does not change your position, unless the player just behind you is close to you. 
%
%\bp
%\wan{4}\wan{6}\tong{7}\tong{7} \rzhong\zhong\zhong~\suo{9}\rsuo{9}\suo{9}~\rtong{4}\tong{2}\tong{3}
%\ep
%
%\section{Don't impose your set of values upon others}

%\section{Defense against the dark arts: Proposed amendments to EMA rules}

%\section{Oka rules}
%It appears that European players are not very familiar with the {\jap Oka} system (there is no {\jap Oka} in EMA tournaments), so let me explain this with an example. Suppose that the players A, B, C, and D hold the following raw points at the end of a game; 39,000, 25,100, 22900, and 13,000. 
%
%\begin{table}[h!]\centering
%\caption{Score calculation at {\jap tenhou}}
%\begin{tabularx}{320pt}{X r r r r}
%\toprule
%Player & Raw score & Before {\jap Uma} & After {\jap Uma} & After {\jap Oka}\\
%\midrule
%A & 39,000 & $9,000$ & $29,000$ & $49,000$\\
%B & 25,100 & $-4,900$ & $5,100$ & $5,100$ \\
%C & 22900 & $-7100$ & $-17100$ & $-17100$\\
%D & 13,000 & $-17,000$ & $-37,000$ & $-37,000$\\
%\bottomrule
%\end{tabularx}
%\end{table}
%
%To illustrate what {\jap Oka} does, let's calculate the scores according to the standard EMA rules. 
%Since every player is allocated 30000 points at the beginning of an EMA game, the score distribution of 39,000, 25,100, 22900, and 13,000 in {\jap tenhou} is equivalent to a score distribution of 44000, 30,100, 27,900, and 18000 in EMA rules. Then, adding the 10-20 {\jap Uma}, we obtain the following. 
%
%\begin{table}[h!]\centering
%\caption{Score calculation at EMA}
%\begin{tabularx}{280pt}{X r r r}
%\toprule
%Player & Raw score & Before {\jap Uma} & After {\jap Uma} \\
%\midrule
%A & 44000 & $14000$ & $34000$  \\
%B & 30,100 & $100$ & $10,100$  \\
%C & 27,900 & $-2,100$ & $-12,100$ \\
%D & 18000 & $-12000$ & $-32000$ \\
%\bottomrule
%\end{tabularx}
%\end{table}
%
%We can see that, even though we start with equivalent raw score distributions, the final score distributions look very different. EMA rules are more egalitarian in the sense that the score differences between the 1st player and the three other players are much smaller. Put differently, getting 1st place is less important in EMA rules than in rules with {\jap Oka}. 
%

